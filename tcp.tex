%----------------------------------------------------------------------------------------
%	PACKAGES AND OTHER DOCUMENT CONFIGURATIONS
%----------------------------------------------------------------------------------------
\documentclass[11pt, a4paper,oneside]{book}

\usepackage{graphicx} % Required for including pictures

%----------------------------------------------------------------------------------------
%       Localization
%----------------------------------------------------------------------------------------
\usepackage[UTF8,adobefonts]{ctex}
\usepackage[TS1,T1]{fontenc}
\usepackage{array, booktabs}
\usepackage{graphicx}
\usepackage[x11names]{xcolor}
\usepackage{colortbl}
\usepackage{fontspec}
\newcommand{\foo}{\color{baseD}\makebox[0pt]{\textbullet}\hskip-0.5pt\vrule width 1pt\hspace{\labelsep}}

%\setmainfont[Boldont=WenQuanYi Micro Hei]{AR PL SungtiL GB}
%\setsansfont[BoldFont=WenQuanYi Micro Hei]{AR PL KaitiM GB}
%\setmonofont{DejaVu Sans Mono}

%\XeTeXlinebreaklocale "zh"
%\XeTeXlinebreakskip = 0pt plus 1pt minus 0.1pt

\usepackage[top=1in,bottom=1in,left=1.25in,right=1.25in]{geometry}
%\linespread{1.2}

\usepackage[Glenn]{fncychap}

\usepackage{fancyhdr}

%----------------------------------------------------------------------------------------
%       Useful Packages
%----------------------------------------------------------------------------------------
\usepackage{color}
\usepackage{url}
\usepackage[colorlinks, linkcolor=black,anchorcolor=black, citecolor=black]{hyperref}

\usepackage{xcolor} % Required for specifying colors by name
\definecolor{ocre}{RGB}{243,102,25} % Define the orange color used for highlighting throughout the book

% BASE16
\definecolor{base0}{HTML}{181818}
\definecolor{base1}{HTML}{282828}
\definecolor{base2}{HTML}{383838}
\definecolor{base3}{HTML}{585858}
\definecolor{base4}{HTML}{B8B8B8}
\definecolor{base5}{HTML}{D8D8D8}
\definecolor{base6}{HTML}{E8E8E8}
\definecolor{base7}{HTML}{F8F8F8}
\definecolor{base8}{HTML}{AB4642}
\definecolor{base9}{HTML}{DC9656}
\definecolor{baseA}{HTML}{F7CA88}
\definecolor{baseB}{HTML}{A1B56C}
\definecolor{baseC}{HTML}{86C1B9}
\definecolor{baseD}{HTML}{7CAFC2}
\definecolor{baseE}{HTML}{BA8BAF}
\definecolor{baseF}{HTML}{A16946}
\definecolor{Gray}{HTML}{CCCCCC}
\definecolor{linkcolor}{HTML}{EC008C}
\definecolor{codecolorpink}{HTML}{CC00FF}
\definecolor{NoteColorFont}{HTML}{6D727D}
\definecolor{NoteColorLine}{HTML}{C3CAD9}
\definecolor{ExeColorFont}{HTML}{FF9900}
\definecolor{ExeColorLine}{HTML}{FFF678}
\definecolor{ExeColorBack}{HTML}{FFFFCC}
\definecolor{ThinkColorFont}{HTML}{629D81}
\definecolor{ThinkColorLine}{HTML}{93E87D}
\definecolor{ThinkColorBack}{HTML}{C1FA9B}

\usepackage{amsmath,amsfonts,amssymb,amsthm} % For math equations, theorems, symbols, etc
\usepackage{booktabs} % For tables
\usepackage{tabularx}
\usepackage{multirow} % for multiple row tables.

%----------------------------------------------------------------------------------------
%       Some Extra Definitions
%----------------------------------------------------------------------------------------

\RequirePackage[framemethod=default]{mdframed} % Required for creating the theorem, definition, exercise and corollary boxes

% Exercise box
\newmdenv[skipabove=10pt,
skipbelow=10pt,
rightline=false,
leftline=true,
topline=false,
bottomline=false,
backgroundcolor=ExeColorBack,
linecolor=ExeColorLine,
innerleftmargin=5pt,
innerrightmargin=5pt,
innertopmargin=5pt,
innerbottommargin=5pt,
leftmargin=0cm,
rightmargin=0cm,
linewidth=12pt]{eBox}

% Thinking box
\newmdenv[skipabove=10pt,
skipbelow=10pt,
rightline=false,
leftline=true,
topline=false,
bottomline=false,
backgroundcolor=ThinkColorBack!30,
linecolor=ThinkColorLine,
innerleftmargin=5pt,
innerrightmargin=5pt,
innertopmargin=5pt,
innerbottommargin=5pt,
leftmargin=0cm,
rightmargin=0cm,
linewidth=12pt]{tBox}

% Note box
\newmdenv[skipabove=10pt,
skipbelow=10pt,
rightline=false,
leftline=true,
topline=false,
bottomline=false,
backgroundcolor=NoteColorLine!15,
linecolor=NoteColorLine,
innerleftmargin=5pt,
innerrightmargin=5pt,
innertopmargin=5pt,
innerbottommargin=5pt,
leftmargin=0cm,
rightmargin=0cm,
linewidth=12pt]{nBox}

% Boxed/framed environments
\newtheoremstyle{ocrenumbox}% % Theorem style name
{0pt}% Space above
{0pt}% Space below
{\normalfont}% % Body font
{}% Indent amount
{\small\bf\sffamily\color{ExeColorFont}}% % Theorem head font
{\;}% Punctuation after theorem head
{0.25em}% Space after theorem head	
{\small\sffamily\color{ExeColorFont}\thmname{#1}\nobreakspace\thmnumber{#2}% Theorem text (e.g. Exercise 2.1)
\thmnote{\nobreakspace\the\thm@notefont\sffamily\bfseries\color{black}---\nobreakspace#3.}} % Optional theorem note
\renewcommand{\qedsymbol}{$\blacksquare$}% Optional qed square

\newtheoremstyle{purplenumbox}% % Theorem style name
{0pt}% Space above
{0pt}% Space below
{\normalfont}% % Body font
{}% Indent amount
{\small\bf\sffamily\color{ThinkColorFont}}% % Theorem head font
{\;}% Punctuation after theorem head
{0.25em}% Space after theorem head	
{\small\sffamily\color{ThinkColorFont}\thmname{#1}\nobreakspace\thmnumber{#2}
% Theorem text (e.g. Thinking 2.1)
\thmnote{\nobreakspace\the\thm@notefont\sffamily\bfseries\color{black}---\nobreakspace#3.}} % Optional theorem note
\renewcommand{\qedsymbol}{$\blacksquare$}% Optional qed square

\newtheoremstyle{blackbox} % Theorem style name
{0pt}% Space above
{0pt}% Space below
{\normalfont}% Body font
{}% Indent amount
{\small\bf\sffamily}% Theorem head font
{\;}% Punctuation after theorem head
{0.25em}% Space after theorem head
{\small\sffamily\color{NoteColorFont}\thmname{#1}\nobreakspace\thmnumber{#2}
% Theorem text (e.g. Theorem 2.1)
\thmnote{\nobreakspace\the\thm@notefont\sffamily\bfseries---\nobreakspace#3.}}% Optional theorem note

% Defines the theorem text style for each type of theorem to one of the three styles above
\theoremstyle{ocrenumbox}
\newtheorem{exerciseT}{Exercise}[chapter]
\theoremstyle{purplenumbox}
\newtheorem{thinkingT}{Thinking}[chapter]
\theoremstyle{blackbox}
\newtheorem{noteT}{Note}[section]

\newenvironment{exercise}{\begin{eBox}\begin{exerciseT}}{\hfill{\color{ExeColorFont}\tiny\ensuremath{\blacksquare}}\end{exerciseT}\end{eBox}}
\newenvironment{thinking}{\begin{tBox}\begin{thinkingT}}{\hfill{\color{ThinkColorFont}\tiny\ensuremath{\blacksquare}}\end{thinkingT}\end{tBox}}
\newenvironment{note}{\begin{nBox}\begin{noteT}}{\end{noteT}\end{nBox}}

%----------------------------------------------------------------------------------------
%       Code Environment
%----------------------------------------------------------------------------------------
\usepackage{minted}
\usemintedstyle{manni}

% code box
\newmdenv[backgroundcolor=base7,
linecolor=baseD,
bottomline=false,
leftline=true,
rightline=false,
topline=false,
linewidth=2pt,
leftmargin=13pt]{pcodeBox}

\renewcommand{\theFancyVerbLine}{
  \sffamily
  \textcolor{baseB}{\arabic{FancyVerbLine}
  }
}

\usepackage{caption}

%\captionsetup{type=codeCaption}
\newenvironment{codeBox}{\begin{pcodeBox}\fontsize{9pt}{9pt}}{\end{pcodeBox}}
\newenvironment{codeBoxWithCaption}[1]{\begin{pcodeBox}[frametitle={\captionof{listing}{#1}\color{base6}\rule{\textwidth}{0.7pt}}]\fontsize{9pt}{9pt}}{\end{pcodeBox}}

\BeforeBeginEnvironment{minted}{\begin{codeBox}}
\AfterEndEnvironment{minted}{\end{codeBox}}

%----------------------------------------------------------------------------------------
%       Lists
%----------------------------------------------------------------------------------------
\usepackage{enumitem}
\setlist[description]{labelindent=22pt} 

%----------------------------------------------------------------------------------------
%       Main Body
%----------------------------------------------------------------------------------------
\begin{document}

\pagestyle{empty} % Removes page numbers
\title{TCP高级实验}
\author{王鹿鸣,刘保证}
\date{\today}
\maketitle
\setcounter{secnumdepth}{3}
\frontmatter
\tableofcontents

\mainmatter
\pagestyle{fancy}

\chapter{网络子系统相关核心数据结构}
	\section{基础数据结构}	
		\subsection{sock\_common}
		\subsection{sock}
			sock结构是比较通用的网络层描述块,构成传输控制块的基础,与具体的协议族无关。它描述了各协议族的公共信息,因此不能直接作为传输层控制块来使用。不同协议族的传输层在使用该结构的时候都会对其进行拓展,来适合各自的传输特性,例如,inet\_sock结构由sock结构及其它特性组成,构成了IPV4协议族传输控制块的基础。结构如下:
\begin{minted}[linenos]{C}
/**
  *	struct sock - network layer representation of sockets
  *	@__sk_common: shared layout with inet_timewait_sock
  *	@sk_shutdown: mask of %SEND_SHUTDOWN and/or %RCV_SHUTDOWN
  *	@sk_userlocks: %SO_SNDBUF and %SO_RCVBUF settings
  *	@sk_lock:	synchronizer
  *	@sk_rcvbuf: size of receive buffer in bytes
  *	@sk_wq: sock wait queue and async head
  *	@sk_rx_dst: receive input route used by early demux
  *	@sk_dst_cache: destination cache
  *	@sk_policy: flow policy
  *	@sk_receive_queue: incoming packets
  *	@sk_wmem_alloc: transmit queue bytes committed
  *	@sk_write_queue: Packet sending queue
  *	@sk_omem_alloc: "o" is "option" or "other"
  *	@sk_wmem_queued: persistent queue size
  *	@sk_forward_alloc: space allocated forward
  *	@sk_napi_id: id of the last napi context to receive data for sk
  *	@sk_ll_usec: usecs to busypoll when there is no data
  *	@sk_allocation: allocation mode
  *	@sk_pacing_rate: Pacing rate (if supported by transport/packet scheduler)
  *	@sk_max_pacing_rate: Maximum pacing rate (%SO_MAX_PACING_RATE)
  *	@sk_sndbuf: size of send buffer in bytes
  *	@sk_no_check_tx: %SO_NO_CHECK setting, set checksum in TX packets
  *	@sk_no_check_rx: allow zero checksum in RX packets
  *	@sk_route_caps: route capabilities (e.g. %NETIF_F_TSO)
  *	@sk_route_nocaps: forbidden route capabilities (e.g NETIF_F_GSO_MASK)
  *	@sk_gso_type: GSO type (e.g. %SKB_GSO_TCPV4)
  *	@sk_gso_max_size: Maximum GSO segment size to build
  *	@sk_gso_max_segs: Maximum number of GSO segments
  *	@sk_lingertime: %SO_LINGER l_linger setting
  *	@sk_backlog: always used with the per-socket spinlock held
  *	@sk_callback_lock: used with the callbacks in the end of this struct
  *	@sk_error_queue: rarely used
  *	@sk_prot_creator: sk_prot of original sock creator (see ipv6_setsockopt,
  *			  IPV6_ADDRFORM for instance)
  *	@sk_err: last error
  *	@sk_err_soft: errors that don't cause failure but are the cause of a
  *		      persistent failure not just 'timed out'
  *	@sk_drops: raw/udp drops counter
  *	@sk_ack_backlog: current listen backlog
  *	@sk_max_ack_backlog: listen backlog set in listen()
  *	@sk_priority: %SO_PRIORITY setting
  *	@sk_cgrp_prioidx: socket group's priority map index
  *	@sk_type: socket type (%SOCK_STREAM, etc)
  *	@sk_protocol: which protocol this socket belongs in this network family
  *	@sk_peer_pid: &struct pid for this socket's peer
  *	@sk_peer_cred: %SO_PEERCRED setting
  *	@sk_rcvlowat: %SO_RCVLOWAT setting
  *	@sk_rcvtimeo: %SO_RCVTIMEO setting
  *	@sk_sndtimeo: %SO_SNDTIMEO setting
  *	@sk_txhash: computed flow hash for use on transmit
  *	@sk_filter: socket filtering instructions
  *	@sk_timer: sock cleanup timer
  *	@sk_stamp: time stamp of last packet received
  *	@sk_tsflags: SO_TIMESTAMPING socket options
  *	@sk_tskey: counter to disambiguate concurrent tstamp requests
  *	@sk_socket: Identd and reporting IO signals
  *	@sk_user_data: RPC layer private data
  *	@sk_frag: cached page frag
  *	@sk_peek_off: current peek_offset value
  *	@sk_send_head: front of stuff to transmit
  *	@sk_security: used by security modules
  *	@sk_mark: generic packet mark
  *	@sk_classid: this socket's cgroup classid
  *	@sk_cgrp: this socket's cgroup-specific proto data
  *	@sk_write_pending: a write to stream socket waits to start
  *	@sk_state_change: callback to indicate change in the state of the sock
  *	@sk_data_ready: callback to indicate there is data to be processed
  *	@sk_write_space: callback to indicate there is bf sending space available
  *	@sk_error_report: callback to indicate errors (e.g. %MSG_ERRQUEUE)
  *	@sk_backlog_rcv: callback to process the backlog
  *	@sk_destruct: called at sock freeing time, i.e. when all refcnt == 0
 */
struct sock {
	/*
	 * Now struct inet_timewait_sock also uses sock_common, so please just
	 * don't add nothing before this first member (__sk_common) --acme
	 */
	struct sock_common	__sk_common;
#define sk_node			__sk_common.skc_node
#define sk_nulls_node		__sk_common.skc_nulls_node
#define sk_refcnt		__sk_common.skc_refcnt
#define sk_tx_queue_mapping	__sk_common.skc_tx_queue_mapping

#define sk_dontcopy_begin	__sk_common.skc_dontcopy_begin
#define sk_dontcopy_end		__sk_common.skc_dontcopy_end
#define sk_hash			__sk_common.skc_hash
#define sk_portpair		__sk_common.skc_portpair
#define sk_num			__sk_common.skc_num
#define sk_dport		__sk_common.skc_dport
#define sk_addrpair		__sk_common.skc_addrpair
#define sk_daddr		__sk_common.skc_daddr
#define sk_rcv_saddr		__sk_common.skc_rcv_saddr
#define sk_family		__sk_common.skc_family
#define sk_state		__sk_common.skc_state
#define sk_reuse		__sk_common.skc_reuse
#define sk_reuseport		__sk_common.skc_reuseport
#define sk_ipv6only		__sk_common.skc_ipv6only
#define sk_net_refcnt		__sk_common.skc_net_refcnt
#define sk_bound_dev_if		__sk_common.skc_bound_dev_if
#define sk_bind_node		__sk_common.skc_bind_node
#define sk_prot			__sk_common.skc_prot
#define sk_net			__sk_common.skc_net
#define sk_v6_daddr		__sk_common.skc_v6_daddr
#define sk_v6_rcv_saddr	__sk_common.skc_v6_rcv_saddr
#define sk_cookie		__sk_common.skc_cookie
#define sk_incoming_cpu		__sk_common.skc_incoming_cpu
#define sk_flags		__sk_common.skc_flags
#define sk_rxhash		__sk_common.skc_rxhash

	socket_lock_t		sk_lock;
	struct sk_buff_head	sk_receive_queue;
	/*
	 * The backlog queue is special, it is always used with
	 * the per-socket spinlock held and requires low latency
	 * access. Therefore we special case it's implementation.
	 * Note : rmem_alloc is in this structure to fill a hole
	 * on 64bit arches, not because its logically part of
	 * backlog.
	 */
	struct {
		atomic_t	rmem_alloc;
		int		len;
		struct sk_buff	*head;
		struct sk_buff	*tail;
	} sk_backlog;
#define sk_rmem_alloc sk_backlog.rmem_alloc
	int			sk_forward_alloc;

	__u32			sk_txhash;
#ifdef CONFIG_NET_RX_BUSY_POLL
	unsigned int		sk_napi_id;
	unsigned int		sk_ll_usec;
#endif
	atomic_t		sk_drops;
	int			sk_rcvbuf;

	struct sk_filter __rcu	*sk_filter;
	union {
		struct socket_wq __rcu	*sk_wq;
		struct socket_wq	*sk_wq_raw;
	};
#ifdef CONFIG_XFRM
	struct xfrm_policy __rcu *sk_policy[2];
#endif
	struct dst_entry	*sk_rx_dst;
	struct dst_entry __rcu	*sk_dst_cache;
	/* Note: 32bit hole on 64bit arches */
	atomic_t		sk_wmem_alloc;
	atomic_t		sk_omem_alloc;
	int			sk_sndbuf;
	struct sk_buff_head	sk_write_queue;
	kmemcheck_bitfield_begin(flags);
	unsigned int		sk_shutdown  : 2,
				sk_no_check_tx : 1,
				sk_no_check_rx : 1,
				sk_userlocks : 4,
				sk_protocol  : 8,
				sk_type      : 16;
#define SK_PROTOCOL_MAX U8_MAX
	kmemcheck_bitfield_end(flags);
	int			sk_wmem_queued;
	gfp_t			sk_allocation;
	u32			sk_pacing_rate; /* bytes per second */
	u32			sk_max_pacing_rate;
	netdev_features_t	sk_route_caps;
	netdev_features_t	sk_route_nocaps;
	int			sk_gso_type;
	unsigned int		sk_gso_max_size;
	u16			sk_gso_max_segs;
	int			sk_rcvlowat;
	unsigned long	        sk_lingertime;
	struct sk_buff_head	sk_error_queue;
	struct proto		*sk_prot_creator;
	rwlock_t		sk_callback_lock;
	int			sk_err,
				sk_err_soft;
	u32			sk_ack_backlog;
	u32			sk_max_ack_backlog;
	__u32			sk_priority;
#if IS_ENABLED(CONFIG_CGROUP_NET_PRIO)
	__u32			sk_cgrp_prioidx;
#endif
	struct pid		*sk_peer_pid;
	const struct cred	*sk_peer_cred;
	long			sk_rcvtimeo;
	long			sk_sndtimeo;
	struct timer_list	sk_timer;
	ktime_t			sk_stamp;
	u16			sk_tsflags;
	u32			sk_tskey;
	struct socket		*sk_socket;
	void			*sk_user_data;
	struct page_frag	sk_frag;
	struct sk_buff		*sk_send_head;
	__s32			sk_peek_off;
	int			sk_write_pending;
#ifdef CONFIG_SECURITY
	void			*sk_security;
#endif
	__u32			sk_mark;
#ifdef CONFIG_CGROUP_NET_CLASSID
	u32			sk_classid;
#endif
	struct cg_proto		*sk_cgrp;
	void			(*sk_state_change)(struct sock *sk);
	void			(*sk_data_ready)(struct sock *sk);
	void			(*sk_write_space)(struct sock *sk);
	void			(*sk_error_report)(struct sock *sk);
	int			(*sk_backlog_rcv)(struct sock *sk,
						  struct sk_buff *skb);
	void                    (*sk_destruct)(struct sock *sk);
};
\end{minted}
	\section{sk\_buff}
	\section{TCP包相关数据结构}
		\subsection{tcphdr /include/uapi/linux/tcp.h}
\begin{minted}[linenos]{C}
	struct tcphdr {
		__be16	source;
		__be16	dest;
		__be32	seq;
		__be32	ack_seq;
	#if defined(__LITTLE_ENDIAN_BITFIELD)
		__u16	res1:4,
			doff:4,
			fin:1,
			syn:1,
			rst:1,
			psh:1,
			ack:1,
			urg:1,
			ece:1,
			cwr:1;
	#elif defined(__BIG_ENDIAN_BITFIELD)
		__u16	doff:4,
			res1:4,
			cwr:1,
			ece:1,
			urg:1,
			ack:1,
			psh:1,
			rst:1,
			syn:1,
			fin:1;
	#else
	#error	"Adjust your <asm/byteorder.h> defines"
	#endif	
		__be16	window;
		__sum16	check;
		__be16	urg_ptr;
	};
\end{minted}
	\section{TCP传输控制块相关数据结构}	
		\subsection{TCP函数接口--inet\_connection\_sock\_af\_ops}
\begin{minted}[linenos]{C}
struct inet_connection_sock_af_ops {
	int	    (*queue_xmit)(struct sock *sk, struct sk_buff *skb, struct flowi *fl);
	void	    (*send_check)(struct sock *sk, struct sk_buff *skb);
	int	    (*rebuild_header)(struct sock *sk);
	void	    (*sk_rx_dst_set)(struct sock *sk, const struct sk_buff *skb);
	int	    (*conn_request)(struct sock *sk, struct sk_buff *skb);
	struct sock *(*syn_recv_sock)(const struct sock *sk, struct sk_buff *skb,
				      struct request_sock *req,
				      struct dst_entry *dst,
				      struct request_sock *req_unhash,
				      bool *own_req);
	u16	    net_header_len;
	u16	    net_frag_header_len;
	u16	    sockaddr_len;
	int	    (*setsockopt)(struct sock *sk, int level, int optname, 
				  char __user *optval, unsigned int optlen);
	int	    (*getsockopt)(struct sock *sk, int level, int optname, 
				  char __user *optval, int __user *optlen);
#ifdef CONFIG_COMPAT
	int	    (*compat_setsockopt)(struct sock *sk,
				int level, int optname,
				char __user *optval, unsigned int optlen);
	int	    (*compat_getsockopt)(struct sock *sk,
				int level, int optname,
				char __user *optval, int __user *optlen);
#endif
	void	    (*addr2sockaddr)(struct sock *sk, struct sockaddr *);
	int	    (*bind_conflict)(const struct sock *sk,
				     const struct inet_bind_bucket *tb, bool relax);
	void	    (*mtu_reduced)(struct sock *sk);
};
\end{minted}	
		\subsection{面向连接传输控制块的表示--inet\_connect\_sock}
\begin{minted}[linenos]{C}
/** inet_connection_sock - INET connection oriented sock
 *
 * @icsk_accept_queue:	   FIFO of established children 
 * @icsk_bind_hash:	   Bind node
 * @icsk_timeout:	   Timeout
 * @icsk_retransmit_timer: Resend (no ack)
 * @icsk_rto:		   Retransmit timeout
 * @icsk_pmtu_cookie	   Last pmtu seen by socket
 * @icsk_ca_ops		   Pluggable congestion control hook
 * @icsk_af_ops		   Operations which are AF_INET{4,6} specific
 * @icsk_ca_state:	   Congestion control state
 * @icsk_retransmits:	   Number of unrecovered [RTO] timeouts
 * @icsk_pending:	   Scheduled timer event
 * @icsk_backoff:	   Backoff
 * @icsk_syn_retries:      Number of allowed SYN (or equivalent) retries
 * @icsk_probes_out:	   unanswered 0 window probes
 * @icsk_ext_hdr_len:	   Network protocol overhead (IP/IPv6 options)
 * @icsk_ack:		   Delayed ACK control data
 * @icsk_mtup;		   MTU probing control data
 */
struct inet_connection_sock {
	/* inet_sock has to be the first member! */
	struct inet_sock	  icsk_inet;
	struct request_sock_queue icsk_accept_queue;
	struct inet_bind_bucket	  *icsk_bind_hash;
	unsigned long		  icsk_timeout;
 	struct timer_list	  icsk_retransmit_timer;
 	struct timer_list	  icsk_delack_timer;
	__u32			  icsk_rto;
	__u32			  icsk_pmtu_cookie;
	const struct tcp_congestion_ops *icsk_ca_ops;
	const struct inet_connection_sock_af_ops *icsk_af_ops;
	unsigned int		  (*icsk_sync_mss)(struct sock *sk, u32 pmtu);
	__u8			  icsk_ca_state:6,
				  icsk_ca_setsockopt:1,
				  icsk_ca_dst_locked:1;
	__u8			  icsk_retransmits;
	__u8			  icsk_pending;
	__u8			  icsk_backoff;
	__u8			  icsk_syn_retries;
	__u8			  icsk_probes_out;
	__u16			  icsk_ext_hdr_len;
	struct {
		__u8		  pending;	 /* ACK is pending			   */
		__u8		  quick;	 /* Scheduled number of quick acks	   */
		__u8		  pingpong;	 /* The session is interactive		   */
		__u8		  blocked;	 /* Delayed ACK was blocked by socket lock */
		__u32		  ato;		 /* Predicted tick of soft clock	   */
		unsigned long	  timeout;	 /* Currently scheduled timeout		   */
		__u32		  lrcvtime;	 /* timestamp of last received data packet */
		__u16		  last_seg_size; /* Size of last incoming segment	   */
		__u16		  rcv_mss;	 /* MSS used for delayed ACK decisions	   */ 
	} icsk_ack;
	struct {
		int		  enabled;

		/* Range of MTUs to search */
		int		  search_high;
		int		  search_low;

		/* Information on the current probe. */
		int		  probe_size;

		u32		  probe_timestamp;
	} icsk_mtup;
	u32			  icsk_user_timeout;

	u64			  icsk_ca_priv[64 / sizeof(u64)];
#define ICSK_CA_PRIV_SIZE      (8 * sizeof(u64))
};
\end{minted}		
		\subsection{连接请求块--tcp\_request\_sock}
\begin{minted}[linenos]{C}
struct tcp_request_sock {
	struct inet_request_sock 	req;
	const struct tcp_request_sock_ops *af_specific;
	struct skb_mstamp		snt_synack; /* first SYNACK sent time */
	bool				tfo_listener;
	u32				txhash;
	u32				rcv_isn;
	u32				snt_isn;
	u32				last_oow_ack_time; /* last SYNACK */
	u32				rcv_nxt; /* the ack # by SYNACK. For
						  * FastOpen it's the seq#
						  * after data-in-SYN.
						  */
};
\end{minted}
		\subsection{TCP协议控制块--tcp\_sock}
\begin{minted}[linenos]{C}
struct tcp_sock {
	/* inet_connection_sock has to be the first member of tcp_sock */
	struct inet_connection_sock	inet_conn;
	u16	tcp_header_len;	/* Bytes of tcp header to send		*/
	u16	gso_segs;	/* Max number of segs per GSO packet	*/

/*
 *	Header prediction flags
 *	0x5?10 << 16 + snd_wnd in net byte order
 */
	__be32	pred_flags;

/*
 *	RFC793 variables by their proper names. This means you can
 *	read the code and the spec side by side (and laugh ...)
 *	See RFC793 and RFC1122. The RFC writes these in capitals.
 */
	u64	bytes_received;	/* RFC4898 tcpEStatsAppHCThruOctetsReceived
				 * sum(delta(rcv_nxt)), or how many bytes
				 * were acked.
				 */
	u32	segs_in;	/* RFC4898 tcpEStatsPerfSegsIn
				 * total number of segments in.
				 */
 	u32	rcv_nxt;	/* What we want to receive next 	*/
	u32	copied_seq;	/* Head of yet unread data		*/
	u32	rcv_wup;	/* rcv_nxt on last window update sent	*/
 	u32	snd_nxt;	/* Next sequence we send		*/
	u32	segs_out;	/* RFC4898 tcpEStatsPerfSegsOut
				 * The total number of segments sent.
				 */
	u64	bytes_acked;	/* RFC4898 tcpEStatsAppHCThruOctetsAcked
				 * sum(delta(snd_una)), or how many bytes
				 * were acked.
				 */
	struct u64_stats_sync syncp; /* protects 64bit vars (cf tcp_get_info()) */

 	u32	snd_una;	/* First byte we want an ack for	*/
 	u32	snd_sml;	/* Last byte of the most recently transmitted small packet */
	u32	rcv_tstamp;	/* timestamp of last received ACK (for keepalives) */
	u32	lsndtime;	/* timestamp of last sent data packet (for restart window) */
	u32	last_oow_ack_time;  /* timestamp of last out-of-window ACK */

	u32	tsoffset;	/* timestamp offset */

	struct list_head tsq_node; /* anchor in tsq_tasklet.head list */
	unsigned long	tsq_flags;

	/* Data for direct copy to user */
	struct {
		struct sk_buff_head	prequeue;
		struct task_struct	*task;
		struct msghdr		*msg;
		int			memory;
		int			len;
	} ucopy;

	u32	snd_wl1;	/* Sequence for window update		*/
	u32	snd_wnd;	/* The window we expect to receive	*/
	u32	max_window;	/* Maximal window ever seen from peer	*/
	u32	mss_cache;	/* Cached effective mss, not including SACKS */

	u32	window_clamp;	/* Maximal window to advertise		*/
	u32	rcv_ssthresh;	/* Current window clamp			*/

	/* Information of the most recently (s)acked skb */
	struct tcp_rack {
		struct skb_mstamp mstamp; /* (Re)sent time of the skb */
		u8 advanced; /* mstamp advanced since last lost marking */
		u8 reord;    /* reordering detected */
	} rack;
	u16	advmss;		/* Advertised MSS			*/
	u8	unused;
	u8	nonagle     : 4,/* Disable Nagle algorithm?             */
		thin_lto    : 1,/* Use linear timeouts for thin streams */
		thin_dupack : 1,/* Fast retransmit on first dupack      */
		repair      : 1,
		frto        : 1;/* F-RTO (RFC5682) activated in CA_Loss */
	u8	repair_queue;
	u8	do_early_retrans:1,/* Enable RFC5827 early-retransmit  */
		syn_data:1,	/* SYN includes data */
		syn_fastopen:1,	/* SYN includes Fast Open option */
		syn_fastopen_exp:1,/* SYN includes Fast Open exp. option */
		syn_data_acked:1,/* data in SYN is acked by SYN-ACK */
		save_syn:1,	/* Save headers of SYN packet */
		is_cwnd_limited:1;/* forward progress limited by snd_cwnd? */
	u32	tlp_high_seq;	/* snd_nxt at the time of TLP retransmit. */

/* RTT measurement */
	u32	srtt_us;	/* smoothed round trip time << 3 in usecs */
	u32	mdev_us;	/* medium deviation			*/
	u32	mdev_max_us;	/* maximal mdev for the last rtt period	*/
	u32	rttvar_us;	/* smoothed mdev_max			*/
	u32	rtt_seq;	/* sequence number to update rttvar	*/
	struct rtt_meas {
		u32 rtt, ts;	/* RTT in usec and sampling time in jiffies. */
	} rtt_min[3];

	u32	packets_out;	/* Packets which are "in flight"	*/
	u32	retrans_out;	/* Retransmitted packets out		*/
	u32	max_packets_out;  /* max packets_out in last window */
	u32	max_packets_seq;  /* right edge of max_packets_out flight */

	u16	urg_data;	/* Saved octet of OOB data and control flags */
	u8	ecn_flags;	/* ECN status bits.			*/
	u8	keepalive_probes; /* num of allowed keep alive probes	*/
	u32	reordering;	/* Packet reordering metric.		*/
	u32	snd_up;		/* Urgent pointer		*/

/*
 *      Options received (usually on last packet, some only on SYN packets).
 */
	struct tcp_options_received rx_opt;

/*
 *	Slow start and congestion control (see also Nagle, and Karn & Partridge)
 */
 	u32	snd_ssthresh;	/* Slow start size threshold		*/
 	u32	snd_cwnd;	/* Sending congestion window		*/
	u32	snd_cwnd_cnt;	/* Linear increase counter		*/
	u32	snd_cwnd_clamp; /* Do not allow snd_cwnd to grow above this */
	u32	snd_cwnd_used;
	u32	snd_cwnd_stamp;
	u32	prior_cwnd;	/* Congestion window at start of Recovery. */
	u32	prr_delivered;	/* Number of newly delivered packets to
				 * receiver in Recovery. */
	u32	prr_out;	/* Total number of pkts sent during Recovery. */

 	u32	rcv_wnd;	/* Current receiver window		*/
	u32	write_seq;	/* Tail(+1) of data held in tcp send buffer */
	u32	notsent_lowat;	/* TCP_NOTSENT_LOWAT */
	u32	pushed_seq;	/* Last pushed seq, required to talk to windows */
	u32	lost_out;	/* Lost packets			*/
	u32	sacked_out;	/* SACK'd packets			*/
	u32	fackets_out;	/* FACK'd packets			*/

	/* from STCP, retrans queue hinting */
	struct sk_buff* lost_skb_hint;
	struct sk_buff *retransmit_skb_hint;

	/* OOO segments go in this list. Note that socket lock must be held,
	 * as we do not use sk_buff_head lock.
	 */
	struct sk_buff_head	out_of_order_queue;

	/* SACKs data, these 2 need to be together (see tcp_options_write) */
	struct tcp_sack_block duplicate_sack[1]; /* D-SACK block */
	struct tcp_sack_block selective_acks[4]; /* The SACKS themselves*/

	struct tcp_sack_block recv_sack_cache[4];

	struct sk_buff *highest_sack;   /* skb just after the highest
					 * skb with SACKed bit set
					 * (validity guaranteed only if
					 * sacked_out > 0)
					 */

	int     lost_cnt_hint;
	u32     retransmit_high;	/* L-bits may be on up to this seqno */

	u32	prior_ssthresh; /* ssthresh saved at recovery start	*/
	u32	high_seq;	/* snd_nxt at onset of congestion	*/

	u32	retrans_stamp;	/* Timestamp of the last retransmit,
				 * also used in SYN-SENT to remember stamp of
				 * the first SYN. */
	u32	undo_marker;	/* snd_una upon a new recovery episode. */
	int	undo_retrans;	/* number of undoable retransmissions. */
	u32	total_retrans;	/* Total retransmits for entire connection */

	u32	urg_seq;	/* Seq of received urgent pointer */
	unsigned int		keepalive_time;	  /* time before keep alive takes place */
	unsigned int		keepalive_intvl;  /* time interval between keep alive probes */

	int			linger2;

/* Receiver side RTT estimation */
	struct {
		u32	rtt;
		u32	seq;
		u32	time;
	} rcv_rtt_est;

/* Receiver queue space */
	struct {
		int	space;
		u32	seq;
		u32	time;
	} rcvq_space;

/* TCP-specific MTU probe information. */
	struct {
		u32		  probe_seq_start;
		u32		  probe_seq_end;
	} mtu_probe;
	u32	mtu_info; /* We received an ICMP_FRAG_NEEDED / ICMPV6_PKT_TOOBIG
			   * while socket was owned by user.
			   */

#ifdef CONFIG_TCP_MD5SIG
/* TCP AF-Specific parts; only used by MD5 Signature support so far */
	const struct tcp_sock_af_ops	*af_specific;

/* TCP MD5 Signature Option information */
	struct tcp_md5sig_info	__rcu *md5sig_info;
#endif

/* TCP fastopen related information */
	struct tcp_fastopen_request *fastopen_req;
	/* fastopen_rsk points to request_sock that resulted in this big
	 * socket. Used to retransmit SYNACKs etc.
	 */
	struct request_sock *fastopen_rsk;
	u32	*saved_syn;
};

\end{minted}
\chapter{TCP建立连接过程}
    \section{TCP主动打开-客户}
    \section{TCP被动打开-服务器}
        \subsection{基本流程}
            tcp想要被动打开,就必须得先进行listen调用\textbf{(什么时候被调用呢?)}。经过listen调用之后,系统内部其实创建了一个监听套接字,专门负责监听是否有数据发来,而不会负责传输数据。

            当客户端的第一个syn包到达服务器时,其实linux 内核并不会创建sock结构体,而是创建一个轻量级的request\_sock 结构体,里面能唯一确定某个客户端发来的syn的信息,接着就发送syn、ack给客户端。

            客户端一般就接着回ack。这时,我们能从ack中,取出信息,在一堆request\_sock匹配,看看是否之前有这个ack对应的syn发过来过。如果之前发过syn,那么现在我们就能找到request\_sock,也就是客户端syn时建立的request\_sock。 此时,我们内核才会为这条流创建sock结构体,毕竟,sock结构体比request\_sock大的多,犯不着三次握手都没建立起来我就建立一个大的结构体。当三次握手建立以后,内核就建立一个相对完整的sock,所谓相对完整,其实也是不完整。因为如果你写过socket程序你就知道,所谓的真正完整,是建立socket,而不是sock (socket 结构体中有一个指针sock * sk,显然sock只是socket的一个子集)。那么我们什么时候才会创建完整的socket,或者换句话说,什么时候使得sock 结构体和文件系统关联从而绑定一个fd,用这个fd就可以用来传输数据呢?所谓fd(file descriptor),一般是BSD Socket的用法,用在Unix/Linux 系统上。在Unix/Linux系统下,一个socket句柄,可以看做是一个文件,在socket上收发数据,相当于对一个文件进行读写,所以一个socket句柄,通常也用表示文件句柄的fd来表示。

            如果你有socket编程经验,那么你一定能想到,那就是在accept系统调用时,返回了一个fd,所以说,是你在accept 时,你三次握手完成后建立的sock才绑定了一个 fd。
        \subsection{第一次握手:接受SYN段}
			\subsubsection{正常的首次握手函数调用概览}
			            
			\subsubsection{LISTEN状态处理接收到的TCP段}
                在进行第一次握手的时候,TCP一般处于LISTEN状态。传输控制块接收处理的段都由tcp\_v4\_do\_rcv来处理。该函数位于/net/ipv4/tcp\_ipv4.c中。该函数会根据不同的TCP状态进行不同的处理,这里我们只是讨论第一次握手的函数处理过程。
\begin{minted}[linenos]{C}
/* The socket must have it's spinlock held when we get
 * here, unless it is a TCP_LISTEN socket.
 *
 * We have a potential double-lock case here, so even when
 * doing backlog processing we use the BH locking scheme.
 * This is because we cannot sleep with the original spinlock
 * held.
 */
int tcp_v4_do_rcv(struct sock *sk, struct sk_buff *skb)
{
	struct sock *rsk;

	/*省略无关代码*/

	if (tcp_checksum_complete(skb))
		goto csum_err;

	if (sk->sk_state == TCP_LISTEN) {
		struct sock *nsk = tcp_v4_cookie_check(sk, skb);

		if (!nsk)
			goto discard;
		if (nsk != sk) {
			sock_rps_save_rxhash(nsk, skb);
			sk_mark_napi_id(nsk, skb);
			if (tcp_child_process(sk, nsk, skb)) {
				rsk = nsk;
				goto reset;
			}
			return 0;
		}
	} else
		sock_rps_save_rxhash(sk, skb);

	if (tcp_rcv_state_process(sk, skb)) {
		rsk = sk;
		goto reset;
	}
	return 0;

reset:
	tcp_v4_send_reset(rsk, skb);
discard:
	kfree_skb(skb);
	/* Be careful here. If this function gets more complicated and
	 * gcc suffers from register pressure on the x86, sk (in \%ebx)
	 * might be destroyed here. This current version compiles correctly,
	 * but you have been warned.
	 */
	return 0;

csum_err:
	TCP_INC_STATS_BH(sock_net(sk), TCP_MIB_CSUMERRORS);
	TCP_INC_STATS_BH(sock_net(sk), TCP_MIB_INERRS);
	goto discard;
}
\end{minted}

                \textbf{函数的参数的意思。表格显示(函数头,函数功能,函数参数及相关简单解释),代码行数放在前面。}
                首先,程序先基于伪首部累加和进行全包的校验和,判断包是否传输正确。

                其次,程序会进行相应的cookie检查。

                最后,程序会继续调用tcp\_rcv\_state\_process函数处理接收到的SYN段。
            
	\subsubsection{LISTEN状态处理请求--tcp\_v4\_cookie\_check}
                该函数如下:
\begin{minted}[linenos]{C}
static struct sock *tcp_v4_cookie_check(struct sock *sk, struct sk_buff *skb)
{
#ifdef CONFIG_SYN_COOKIES
	const struct tcphdr *th = tcp_hdr(skb);

	if (!th->syn)
		sk = cookie_v4_check(sk, skb);
#endif
	return sk;
}
\end{minted}

                可以看出如果系统定义了CONFIG\_SYN\_COOKIES宏的话,并且当前并不是syn包,内核就会继续进行cookie\_v4\_check,否则返回sk。显然对于第一次握手的时候,接收到的确实是syn包,故而不会进行检查。而是直接返回了sk。对于cookie\_v4\_check函数,当内存不足时,就返回NULL,否则就返回sk。
            \subsubsection{LISTEN状态处理SYN段--tcp\_rcv\_state\_process}
                该函数位于/net/ipv4/tcp\_input.c中。函数的简要介绍如下:

                与第一次握手相关的代码如下:

\begin{minted}[linenos]{C}
/*
 *	This function implements the receiving procedure of RFC 793 for
 *	all states except ESTABLISHED and TIME_WAIT.
 *	It's called from both tcp_v4_rcv and tcp_v6_rcv and should be
 *	address independent.
 */

int tcp_rcv_state_process(struct sock *sk, struct sk_buff *skb)
{
	struct tcp_sock *tp = tcp_sk(sk);
	struct inet_connection_sock *icsk = inet_csk(sk);
	const struct tcphdr *th = tcp_hdr(skb);
	struct request_sock *req;
	int queued = 0;
	bool acceptable;

	tp->rx_opt.saw_tstamp = 0;

	switch (sk->sk_state) {
	/*省略无关代码*/

	case TCP_LISTEN:
		if (th->ack)
			return 1;

		if (th->rst)
			goto discard;

		if (th->syn) {
			if (th->fin)
				goto discard;
			if (icsk->icsk_af_ops->conn_request(sk, skb) < 0)
				return 1;

			/* Now we have several options: In theory there is
			 * nothing else in the frame. KA9Q has an option to
			 * send data with the syn, BSD accepts data with the
			 * syn up to the [to be] advertised window and
			 * Solaris 2.1 gives you a protocol error. For now
			 * we just ignore it, that fits the spec precisely
			 * and avoids incompatibilities. It would be nice in
			 * future to drop through and process the data.
			 *
			 * Now that TTCP is starting to be used we ought to
			 * queue this data.
			 * But, this leaves one open to an easy denial of
			 * service attack, and SYN cookies can't defend
			 * against this problem. So, we drop the data
			 * in the interest of security over speed unless
			 * it's still in use.
			 */
			kfree_skb(skb);
			return 0;
		}
		goto discard;

		/*省略无关代码*/
discard:
		__kfree_skb(skb);
	}
	return 0;
}
\end{minted}

                显然,所接收到的包的ack、rst、fin字段都不为1,故而执行??行程序。这时开始进行连接检查,判断是否可以允许连接。\textbf{经过不断查找},我们可以发现最终会掉用tcp\_v4\_conn\_request进行处理。如果syn段合法,内核就会为该连接请求创建连接请求块,并且保存相应的信息。否则,就会返回1,原函数会发送reset给客户端表明连接请求失败。

				当然,如果收到的包的ack字段为1,那么由于此时链接还未建立,故该包无效,返回1,并且调用该函数的函数会发送reset包给对方。如果收到的是rst字段或者既有fin又有syn的字段,那就直接销毁,并且释放内存。
            \subsubsection{连接请求处理--tcp\_v4\_conn\_request  tcp\_conn\_request}
				该函数位于/net/ipv4/tcp\_ipv4/tcp\_ipv4.c中,该函数如下:
\begin{minted}[linenos]{C}
int tcp_v4_conn_request(struct sock *sk, struct sk_buff *skb)
{
	/* Never answer to SYNs send to broadcast or multicast */
	if (skb_rtable(skb)->rt_flags & (RTCF_BROADCAST | RTCF_MULTICAST))
		goto drop;

	return tcp_conn_request(&tcp_request_sock_ops,
				&tcp_request_sock_ipv4_ops, sk, skb);

drop:
	NET_INC_STATS_BH(sock_net(sk), LINUX_MIB_LISTENDROPS);
	return 0;
}
\end{minted}
				首先,如果一个SYN段是要被发送到广播地址和组播地址,则直接drop掉,然后返回0。否则的话,就继续调用tcp\_conn\_request进行连接处理。
\begin{minted}[linenos]{C}
int tcp_conn_request(struct request_sock_ops *rsk_ops,
		     const struct tcp_request_sock_ops *af_ops,
		     struct sock *sk, struct sk_buff *skb)
{
	struct tcp_fastopen_cookie foc = { .len = -1 };
	__u32 isn = TCP_SKB_CB(skb)->tcp_tw_isn;
	struct tcp_options_received tmp_opt;
	struct tcp_sock *tp = tcp_sk(sk);
	struct sock *fastopen_sk = NULL;
	struct dst_entry *dst = NULL;
	struct request_sock *req;
	bool want_cookie = false;
	struct flowi fl;

	/* TW buckets are converted to open requests without
	 * limitations, they conserve resources and peer is
	 * evidently real one.
	 */
	if ((sysctl_tcp_syncookies == 2 ||
	     inet_csk_reqsk_queue_is_full(sk)) && !isn) {
		want_cookie = tcp_syn_flood_action(sk, skb, rsk_ops->slab_name);
		if (!want_cookie)
			goto drop;
	}
\end{minted}
				首先,前面???如果SYN请求队列已满并且isn为0,需要查看是否
		
\end{document}
