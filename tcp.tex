%----------------------------------------------------------------------------------------
%	PACKAGES AND OTHER DOCUMENT CONFIGURATIONS
%----------------------------------------------------------------------------------------
\documentclass[11pt, a4paper,oneside]{book}

\usepackage{graphicx} % Required for including pictures

%----------------------------------------------------------------------------------------
%       Localization
%----------------------------------------------------------------------------------------
\usepackage[UTF8,adobefonts]{ctex}
\usepackage[TS1,T1]{fontenc}
\usepackage{array, booktabs}
\usepackage{graphicx}
\usepackage[x11names]{xcolor}
\usepackage{colortbl}
\usepackage{fontspec}
\newcommand{\foo}{\color{baseD}\makebox[0pt]{\textbullet}\hskip-0.5pt\vrule width 1pt\hspace{\labelsep}}

%\setmainfont[Boldont=WenQuanYi Micro Hei]{AR PL SungtiL GB}
%\setsansfont[BoldFont=WenQuanYi Micro Hei]{AR PL KaitiM GB}
%\setmonofont{DejaVu Sans Mono}

%\XeTeXlinebreaklocale "zh"
%\XeTeXlinebreakskip = 0pt plus 1pt minus 0.1pt

\usepackage[top=1in,bottom=1in,left=1.25in,right=1.25in]{geometry}
%\linespread{1.2}

\usepackage[Glenn]{fncychap}

\usepackage{fancyhdr}

%----------------------------------------------------------------------------------------
%       Useful Packages
%----------------------------------------------------------------------------------------
\usepackage{color}
\usepackage{url}
\usepackage[colorlinks, linkcolor=black,anchorcolor=black, citecolor=black]{hyperref}

\usepackage{xcolor} % Required for specifying colors by name
\definecolor{ocre}{RGB}{243,102,25} % Define the orange color used for highlighting throughout the book

% BASE16
\definecolor{base0}{HTML}{181818}
\definecolor{base1}{HTML}{282828}
\definecolor{base2}{HTML}{383838}
\definecolor{base3}{HTML}{585858}
\definecolor{base4}{HTML}{B8B8B8}
\definecolor{base5}{HTML}{D8D8D8}
\definecolor{base6}{HTML}{E8E8E8}
\definecolor{base7}{HTML}{F8F8F8}
\definecolor{base8}{HTML}{AB4642}
\definecolor{base9}{HTML}{DC9656}
\definecolor{baseA}{HTML}{F7CA88}
\definecolor{baseB}{HTML}{A1B56C}
\definecolor{baseC}{HTML}{86C1B9}
\definecolor{baseD}{HTML}{7CAFC2}
\definecolor{baseE}{HTML}{BA8BAF}
\definecolor{baseF}{HTML}{A16946}
\definecolor{Gray}{HTML}{CCCCCC}
\definecolor{linkcolor}{HTML}{EC008C}
\definecolor{codecolorpink}{HTML}{CC00FF}
\definecolor{NoteColorFont}{HTML}{6D727D}
\definecolor{NoteColorLine}{HTML}{C3CAD9}
\definecolor{ExeColorFont}{HTML}{FF9900}
\definecolor{ExeColorLine}{HTML}{FFF678}
\definecolor{ExeColorBack}{HTML}{FFFFCC}
\definecolor{ThinkColorFont}{HTML}{629D81}
\definecolor{ThinkColorLine}{HTML}{93E87D}
\definecolor{ThinkColorBack}{HTML}{C1FA9B}

\usepackage{amsmath,amsfonts,amssymb,amsthm} % For math equations, theorems, symbols, etc
\usepackage{booktabs} % For tables
\usepackage{tabularx}
\usepackage{multirow} % for multiple row tables.

%----------------------------------------------------------------------------------------
%       Some Extra Definitions
%----------------------------------------------------------------------------------------

\RequirePackage[framemethod=default]{mdframed} % Required for creating the theorem, definition, exercise and corollary boxes

% Exercise box
\newmdenv[skipabove=10pt,
skipbelow=10pt,
rightline=false,
leftline=true,
topline=false,
bottomline=false,
backgroundcolor=ExeColorBack,
linecolor=ExeColorLine,
innerleftmargin=5pt,
innerrightmargin=5pt,
innertopmargin=5pt,
innerbottommargin=5pt,
leftmargin=0cm,
rightmargin=0cm,
linewidth=12pt]{eBox}

% Thinking box
\newmdenv[skipabove=10pt,
skipbelow=10pt,
rightline=false,
leftline=true,
topline=false,
bottomline=false,
backgroundcolor=ThinkColorBack!30,
linecolor=ThinkColorLine,
innerleftmargin=5pt,
innerrightmargin=5pt,
innertopmargin=5pt,
innerbottommargin=5pt,
leftmargin=0cm,
rightmargin=0cm,
linewidth=12pt]{tBox}

% Note box
\newmdenv[skipabove=10pt,
skipbelow=10pt,
rightline=false,
leftline=true,
topline=false,
bottomline=false,
backgroundcolor=NoteColorLine!15,
linecolor=NoteColorLine,
innerleftmargin=5pt,
innerrightmargin=5pt,
innertopmargin=5pt,
innerbottommargin=5pt,
leftmargin=0cm,
rightmargin=0cm,
linewidth=12pt]{nBox}

% Boxed/framed environments
\newtheoremstyle{ocrenumbox}% % Theorem style name
{0pt}% Space above
{0pt}% Space below
{\normalfont}% % Body font
{}% Indent amount
{\small\bf\sffamily\color{ExeColorFont}}% % Theorem head font
{\;}% Punctuation after theorem head
{0.25em}% Space after theorem head	
{\small\sffamily\color{ExeColorFont}\thmname{#1}\nobreakspace\thmnumber{#2}% Theorem text (e.g. Exercise 2.1)
\thmnote{\nobreakspace\the\thm@notefont\sffamily\bfseries\color{black}---\nobreakspace#3.}} % Optional theorem note
\renewcommand{\qedsymbol}{$\blacksquare$}% Optional qed square

\newtheoremstyle{purplenumbox}% % Theorem style name
{0pt}% Space above
{0pt}% Space below
{\normalfont}% % Body font
{}% Indent amount
{\small\bf\sffamily\color{ThinkColorFont}}% % Theorem head font
{\;}% Punctuation after theorem head
{0.25em}% Space after theorem head	
{\small\sffamily\color{ThinkColorFont}\thmname{#1}\nobreakspace\thmnumber{#2}
% Theorem text (e.g. Thinking 2.1)
\thmnote{\nobreakspace\the\thm@notefont\sffamily\bfseries\color{black}---\nobreakspace#3.}} % Optional theorem note
\renewcommand{\qedsymbol}{$\blacksquare$}% Optional qed square

\newtheoremstyle{blackbox} % Theorem style name
{0pt}% Space above
{0pt}% Space below
{\normalfont}% Body font
{}% Indent amount
{\small\bf\sffamily}% Theorem head font
{\;}% Punctuation after theorem head
{0.25em}% Space after theorem head
{\small\sffamily\color{NoteColorFont}\thmname{#1}\nobreakspace\thmnumber{#2}
% Theorem text (e.g. Theorem 2.1)
\thmnote{\nobreakspace\the\thm@notefont\sffamily\bfseries---\nobreakspace#3.}}% Optional theorem note

% Defines the theorem text style for each type of theorem to one of the three styles above
\theoremstyle{ocrenumbox}
\newtheorem{exerciseT}{Exercise}[chapter]
\theoremstyle{purplenumbox}
\newtheorem{thinkingT}{Thinking}[chapter]
\theoremstyle{blackbox}
\newtheorem{noteT}{Note}[section]

\newenvironment{exercise}{\begin{eBox}\begin{exerciseT}}{\hfill{\color{ExeColorFont}\tiny\ensuremath{\blacksquare}}\end{exerciseT}\end{eBox}}
\newenvironment{thinking}{\begin{tBox}\begin{thinkingT}}{\hfill{\color{ThinkColorFont}\tiny\ensuremath{\blacksquare}}\end{thinkingT}\end{tBox}}
\newenvironment{note}{\begin{nBox}\begin{noteT}}{\end{noteT}\end{nBox}}

%----------------------------------------------------------------------------------------
%       Code Environment
%----------------------------------------------------------------------------------------
\usepackage{minted}
\usemintedstyle{manni}

% code box
\newmdenv[backgroundcolor=base7,
linecolor=baseD,
bottomline=false,
leftline=true,
rightline=false,
topline=false,
linewidth=2pt,
leftmargin=13pt]{pcodeBox}

\renewcommand{\theFancyVerbLine}{
  \sffamily
  \textcolor{baseB}{\arabic{FancyVerbLine}
  }
}

\usepackage{caption}

%\captionsetup{type=codeCaption}
\newenvironment{codeBox}{\begin{pcodeBox}\fontsize{9pt}{9pt}}{\end{pcodeBox}}
\newenvironment{codeBoxWithCaption}[1]{\begin{pcodeBox}[frametitle={\captionof{listing}{#1}\color{base6}\rule{\textwidth}{0.7pt}}]\fontsize{9pt}{9pt}}{\end{pcodeBox}}

\BeforeBeginEnvironment{minted}{\begin{codeBox}}
\AfterEndEnvironment{minted}{\end{codeBox}}

%----------------------------------------------------------------------------------------
%       Lists
%----------------------------------------------------------------------------------------
\usepackage{enumitem}
\setlist[description]{labelindent=22pt} 

%----------------------------------------------------------------------------------------
%       Main Body
%----------------------------------------------------------------------------------------
\begin{document}

\pagestyle{empty} % Removes page numbers
\title{TCP高级实验}
\author{王鹿鸣,刘保证}
\date{\today}
\maketitle
\setcounter{secnumdepth}{3}
\frontmatter
\tableofcontents

\mainmatter
\pagestyle{fancy}

\chapter{重要的数据结构}
		icsk
		sock\_common
		sock
		sk\_buff
		
\chapter{TCP建立连接过程}
    \section{TCP主动打开-客户}
    \section{TCP被动打开-服务器}
        \subsection{基本流程}
            tcp想要被动打开,就必须得先进行listen调用\textbf{(什么时候被调用呢?)}。经过listen调用之后,系统内部其实创建了一个监听套接字,专门负责监听是否有数据发来,而不会负责传输数据。

            当客户端的第一个syn包到达服务器时,其实linux 内核并不会创建sock结构体,而是创建一个轻量级的request\_sock 结构体,里面能唯一确定某个客户端发来的syn的信息,接着就发送syn、ack给客户端。

            客户端一般就接着回ack。这时,我们能从ack中,取出信息,在一堆request\_sock匹配,看看是否之前有这个ack对应的syn发过来过。如果之前发过syn,那么现在我们就能找到request\_sock,也就是客户端syn时建立的request\_sock。 此时,我们内核才会为这条流创建sock结构体,毕竟,sock结构体比request\_sock大的多,犯不着三次握手都没建立起来我就建立一个大的结构体。当三次握手建立以后,内核就建立一个相对完整的sock,所谓相对完整,其实也是不完整。因为如果你写过socket程序你就知道,所谓的真正完整,是建立socket,而不是sock (socket 结构体中有一个指针sock * sk,显然sock只是socket的一个子集)。那么我们什么时候才会创建完整的socket,或者换句话说,什么时候使得sock 结构体和文件系统关联从而绑定一个fd,用这个fd就可以用来传输数据呢?所谓fd(file descriptor),一般是BSD Socket的用法,用在Unix/Linux 系统上。在Unix/Linux系统下,一个socket句柄,可以看做是一个文件,在socket上收发数据,相当于对一个文件进行读写,所以一个socket句柄,通常也用表示文件句柄的fd来表示。

            如果你有socket编程经验,那么你一定能想到,那就是在accept系统调用时,返回了一个fd,所以说,是你在accept 时,你三次握手完成后建立的sock才绑定了一个 fd。
        \subsection{第一次握手:接受SYN段}
			\subsubsection{正常的首次握手函数调用概览}
			            
			\subsubsection{LISTEN状态处理接收到的TCP段}
                在进行第一次握手的时候,TCP一般处于LISTEN状态。传输控制块接收处理的段都由tcp\_v4\_do\_rcv来处理。该函数位于/net/ipv4/tcp\_ipv4.c中。该函数会根据不同的TCP状态进行不同的处理,这里我们只是讨论第一次握手的函数处理过程。
\begin{minted}[linenos]{C}
/* The socket must have it's spinlock held when we get
 * here, unless it is a TCP_LISTEN socket.
 *
 * We have a potential double-lock case here, so even when
 * doing backlog processing we use the BH locking scheme.
 * This is because we cannot sleep with the original spinlock
 * held.
 */
int tcp_v4_do_rcv(struct sock *sk, struct sk_buff *skb)
{
	struct sock *rsk;

	/*省略无关代码*/

	if (tcp_checksum_complete(skb))
		goto csum_err;

	if (sk->sk_state == TCP_LISTEN) {
		struct sock *nsk = tcp_v4_cookie_check(sk, skb);

		if (!nsk)
			goto discard;
		if (nsk != sk) {
			sock_rps_save_rxhash(nsk, skb);
			sk_mark_napi_id(nsk, skb);
			if (tcp_child_process(sk, nsk, skb)) {
				rsk = nsk;
				goto reset;
			}
			return 0;
		}
	} else
		sock_rps_save_rxhash(sk, skb);

	if (tcp_rcv_state_process(sk, skb)) {
		rsk = sk;
		goto reset;
	}
	return 0;

reset:
	tcp_v4_send_reset(rsk, skb);
discard:
	kfree_skb(skb);
	/* Be careful here. If this function gets more complicated and
	 * gcc suffers from register pressure on the x86, sk (in \%ebx)
	 * might be destroyed here. This current version compiles correctly,
	 * but you have been warned.
	 */
	return 0;

csum_err:
	TCP_INC_STATS_BH(sock_net(sk), TCP_MIB_CSUMERRORS);
	TCP_INC_STATS_BH(sock_net(sk), TCP_MIB_INERRS);
	goto discard;
}
\end{minted}

                \textbf{函数的参数的意思。表格显示(函数头,函数功能,函数参数及相关简单解释),代码行数放在前面。}
                首先,程序先基于伪首部累加和进行全包的校验和,判断包是否传输正确。

                其次,程序会进行相应的cookie检查。

                最后,程序会继续调用tcp\_rcv\_state\_process函数处理接收到的SYN段。
            
	\subsubsection{LISTEN状态处理请求--tcp\_v4\_cookie\_check}
                该函数如下:
\begin{minted}[linenos]{C}
static struct sock *tcp_v4_cookie_check(struct sock *sk, struct sk_buff *skb)
{
#ifdef CONFIG_SYN_COOKIES
	const struct tcphdr *th = tcp_hdr(skb);

	if (!th->syn)
		sk = cookie_v4_check(sk, skb);
#endif
	return sk;
}
\end{minted}

                可以看出如果系统定义了CONFIG\_SYN\_COOKIES宏的话,并且当前并不是syn包,内核就会继续进行cookie\_v4\_check,否则返回sk。显然对于第一次握手的时候,接收到的确实是syn包,故而不会进行检查。而是直接返回了sk。对于cookie\_v4\_check函数,当内存不足时,就返回NULL,否则就返回sk。
            \subsubsection{LISTEN状态处理SYN段--tcp\_rcv\_state\_process}
                该函数位于/net/ipv4/tcp\_input.c中。函数的简要介绍如下:

                与第一次握手相关的代码如下:

\begin{minted}[linenos]{C}
/*
 *	This function implements the receiving procedure of RFC 793 for
 *	all states except ESTABLISHED and TIME_WAIT.
 *	It's called from both tcp_v4_rcv and tcp_v6_rcv and should be
 *	address independent.
 */

int tcp_rcv_state_process(struct sock *sk, struct sk_buff *skb)
{
	struct tcp_sock *tp = tcp_sk(sk);
	struct inet_connection_sock *icsk = inet_csk(sk);
	const struct tcphdr *th = tcp_hdr(skb);
	struct request_sock *req;
	int queued = 0;
	bool acceptable;

	tp->rx_opt.saw_tstamp = 0;

	switch (sk->sk_state) {
	/*省略无关代码*/

	case TCP_LISTEN:
		if (th->ack)
			return 1;

		if (th->rst)
			goto discard;

		if (th->syn) {
			if (th->fin)
				goto discard;
			if (icsk->icsk_af_ops->conn_request(sk, skb) < 0)
				return 1;

			/* Now we have several options: In theory there is
			 * nothing else in the frame. KA9Q has an option to
			 * send data with the syn, BSD accepts data with the
			 * syn up to the [to be] advertised window and
			 * Solaris 2.1 gives you a protocol error. For now
			 * we just ignore it, that fits the spec precisely
			 * and avoids incompatibilities. It would be nice in
			 * future to drop through and process the data.
			 *
			 * Now that TTCP is starting to be used we ought to
			 * queue this data.
			 * But, this leaves one open to an easy denial of
			 * service attack, and SYN cookies can't defend
			 * against this problem. So, we drop the data
			 * in the interest of security over speed unless
			 * it's still in use.
			 */
			kfree_skb(skb);
			return 0;
		}
		goto discard;

		/*省略无关代码*/
discard:
		__kfree_skb(skb);
	}
	return 0;
}
\end{minted}

                显然,所接收到的包的ack、rst、fin字段都不为1,故而执行??行程序。这时开始进行连接检查,判断是否可以允许连接。\textbf{经过不断查找},我们可以发现最终会掉用tcp\_v4\_conn\_request进行处理。如果syn段合法,内核就会为该连接请求创建连接请求块,并且保存相应的信息。否则,就会返回1,原函数会发送reset给客户端表明连接请求失败。

				当然,如果收到的包的ack字段为1,那么由于此时链接还未建立,故该包无效,返回1,并且调用该函数的函数会发送reset包给对方。如果收到的是rst字段或者既有fin又有syn的字段,那就直接销毁,并且释放内存。
            \subsubsection{连接请求处理--tcp\_v4\_conn\_request  tcp\_conn\_request}
				该函数位于/net/ipv4/tcp\_ipv4/tcp\_ipv4.c中,该函数如下:
\begin{minted}[linenos]{C}
int tcp_v4_conn_request(struct sock *sk, struct sk_buff *skb)
{
	/* Never answer to SYNs send to broadcast or multicast */
	if (skb_rtable(skb)->rt_flags & (RTCF_BROADCAST | RTCF_MULTICAST))
		goto drop;

	return tcp_conn_request(&tcp_request_sock_ops,
				&tcp_request_sock_ipv4_ops, sk, skb);

drop:
	NET_INC_STATS_BH(sock_net(sk), LINUX_MIB_LISTENDROPS);
	return 0;
}
\end{minted}
				首先,如果一个SYN段是要被发送到广播地址和组播地址,则直接drop掉,然后返回0。否则的话,就继续调用tcp\_conn\_request进行连接处理。
\begin{minted}[linenos]{C}
int tcp_conn_request(struct request_sock_ops *rsk_ops,
		     const struct tcp_request_sock_ops *af_ops,
		     struct sock *sk, struct sk_buff *skb)
{
	struct tcp_fastopen_cookie foc = { .len = -1 };
	__u32 isn = TCP_SKB_CB(skb)->tcp_tw_isn;
	struct tcp_options_received tmp_opt;
	struct tcp_sock *tp = tcp_sk(sk);
	struct sock *fastopen_sk = NULL;
	struct dst_entry *dst = NULL;
	struct request_sock *req;
	bool want_cookie = false;
	struct flowi fl;

	/* TW buckets are converted to open requests without
	 * limitations, they conserve resources and peer is
	 * evidently real one.
	 */
	if ((sysctl_tcp_syncookies == 2 ||
	     inet_csk_reqsk_queue_is_full(sk)) && !isn) {
		want_cookie = tcp_syn_flood_action(sk, skb, rsk_ops->slab_name);
		if (!want_cookie)
			goto drop;
	}
\end{minted}
				首先,前面???如果SYN请求队列已满并且isn为0,需要查看是否
		
\end{document}
