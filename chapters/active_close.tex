\section{主动关闭}
\label{sec:tcp_active_close}

\subsection{第一次握手——发送FIN}
\label{subsec:tcp_shutdown}
通过shutdown系统调用,主动关闭TCP连接。该系统调用最终由\mintinline{c}{tcp_shutdown}实现。
代码如下:

\begin{minted}[linenos]{c}
void tcp_shutdown(struct sock *sk, int how)
{
        /*      We need to grab some memory, and put together a FIN,
         *      and then put it into the queue to be sent.
         *              Tim MacKenzie(tym@dibbler.cs.monash.edu.au) 4 Dec '92.
         */
        if (!(how & SEND_SHUTDOWN))
                return;

        /* 如果此时已经发送一个FIN了,就跳过。 */
        if ((1 << sk->sk_state) &
            (TCPF_ESTABLISHED | TCPF_SYN_SENT |
             TCPF_SYN_RECV | TCPF_CLOSE_WAIT)) {
                /* Clear out any half completed packets.  FIN if needed. */
                if (tcp_close_state(sk))
                        tcp_send_fin(sk);
        }
}
\end{minted}
该函数会在需要发送FIN时,调用\mintinline{c}{tcp_close_state()}来设置TCP的状态。
该函数会根据当前的状态,按照\ref{subsubsec:tcp_state_diagram}中给出的状态图。
\begin{minted}[linenos]{c}
static const unsigned char new_state[16] = {
  /* 当前状态:             新的状态:        动作:      */
  [0 /* (Invalid) */]   = TCP_CLOSE,
  [TCP_ESTABLISHED]     = TCP_FIN_WAIT1 | TCP_ACTION_FIN,
  [TCP_SYN_SENT]        = TCP_CLOSE,
  [TCP_SYN_RECV]        = TCP_FIN_WAIT1 | TCP_ACTION_FIN,
  [TCP_FIN_WAIT1]       = TCP_FIN_WAIT1,
  [TCP_FIN_WAIT2]       = TCP_FIN_WAIT2,
  [TCP_TIME_WAIT]       = TCP_CLOSE,
  [TCP_CLOSE]           = TCP_CLOSE,
  [TCP_CLOSE_WAIT]      = TCP_LAST_ACK  | TCP_ACTION_FIN,
  [TCP_LAST_ACK]        = TCP_LAST_ACK,
  [TCP_LISTEN]          = TCP_CLOSE,
  [TCP_CLOSING]         = TCP_CLOSING,
  [TCP_NEW_SYN_RECV]    = TCP_CLOSE,    /* should not happen ! */
};

static int tcp_close_state(struct sock *sk)
{
        int next = (int)new_state[sk->sk_state];
        int ns = next & TCP_STATE_MASK;

        /* 根据状态图进行状态转移 */
        tcp_set_state(sk, ns);

        /* 如果需要执行发送FIN的动作,则返回真 */
        return next & TCP_ACTION_FIN;
}
\end{minted}
可以看出,只有当当前状态为TCP\_ESTABLISHED、TCP\_SYN\_RECV、TCP\_CLOSE\_WAIT时,
需要发送FIN包。这个也和TCP状态图一致。如果需要发送FIN包,则会调用
\mintinline{c}{tcp_send_fin}。
\begin{minted}[linenos]{c}
void tcp_send_fin(struct sock *sk)
{
        struct sk_buff *skb, *tskb = tcp_write_queue_tail(sk);
        struct tcp_sock *tp = tcp_sk(sk);

        /* 这里做了一些优化。如果发送队列的末尾还有段没有发出去,则利用该段发送FIN。 */
        if (tskb && (tcp_send_head(sk) || tcp_under_memory_pressure(sk))) {
          /* 如果当前正在发送的队列不为空,或者当前TCP处于内存压力下,则进行该优化 */
coalesce:
                TCP_SKB_CB(tskb)->tcp_flags |= TCPHDR_FIN;
                TCP_SKB_CB(tskb)->end_seq++;
                tp->write_seq++;
                if (!tcp_send_head(sk)) {
                        /* This means tskb was already sent.
                         * Pretend we included the FIN on previous transmit.
                         * We need to set tp->snd_nxt to the value it would have
                         * if FIN had been sent. This is because retransmit path
                         * does not change tp->snd_nxt.
                         */
                        tp->snd_nxt++;
                        return;
                }
        } else {
                /* 为封包分配空间 */
                skb = alloc_skb_fclone(MAX_TCP_HEADER, sk->sk_allocation);
                if (unlikely(!skb)) {
                        /* 如果分配不到空间,且队尾还有未发送的包,利用该包发出FIN。 */
                        if (tskb)
                                goto coalesce;
                        return;
                }
                skb_reserve(skb, MAX_TCP_HEADER);
                sk_forced_mem_schedule(sk, skb->truesize);
                /* FIN eats a sequence byte, write_seq advanced by tcp_queue_skb(). */
                /* 构造一个FIN包,并加入发送队列。 */
                tcp_init_nondata_skb(skb, tp->write_seq,
                                     TCPHDR_ACK | TCPHDR_FIN);
                tcp_queue_skb(sk, skb);
        }
        __tcp_push_pending_frames(sk, tcp_current_mss(sk), TCP_NAGLE_OFF);
}
\end{minted}
在函数的最后,将所有的剩余数据一口气发出去,完成发送FIN包的过程。至此,主动关闭过程的
第一次握手完成。

\subsection{FIN\_WAIT1}

\subsection{FIN\_WAIT2}

\subsection{CLOSING}

\subsection{TIME\_WAIT}
