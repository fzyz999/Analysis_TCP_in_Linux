\section{拥塞控制实现}

	\subsection{拥塞控制状态机}
		\subsubsection{基本转换}

		\subsubsection{Open状态}
			Open状态是常态,在这种状态下TCP发送方通过优化后的快速路径来处理接收ACK。当一个确认到达时,发送方根据拥塞窗口时小于还是大于慢启动阙值,按慢启动或者拥塞避免来增大拥塞窗口。

		\subsubsection{Disorder状态}
			当发送方检测到DACK(重复确认)或者SACK(选择性确认)时,将转变为Disorder(无序)状态。在该状态下,拥塞窗口不做调整,而是每个新到的段出发一个新的数据段的发送。因此,TCP发送方遵循包守恒原则,该原则规定一个新包只有在一个老的包离开网络后才发送。在实践中该规定的表现类似于IETF的传输提议,允许当拥塞窗口较小或是上个传输窗口中有大量数据段丢失时,使用快速重传以更有效地恢复。

		\subsubsection{CWR状态}
			TCP发送方可能从显示拥塞通知、ICMP源端抑制(ICMP source quench)或是本地设备接收到拥塞通知。当收到一个拥塞通知时,发送方并不立刻减小拥塞窗口,而是每隔一个新到的ACK减小一个段直到窗口的大小减半为止。发送方在减小拥塞窗口大小的过程中不会有明显的重传,这就处于CWR(Congestion Window Reduced,拥塞窗口减小)状态。CWR状态可以被Revcovery状态或者Loss状态中断。进入拥塞窗口减小的函数如下:

\begin{minted}[linenos]{C}
/* 
Location:

	net/ipv4/tcp_input.c

Function:

	Enter CWR state. Disable cwnd undo since congestion is proven with ECN 

Parameter:

	sk:传输控制块
*/
void tcp_enter_cwr(struct sock *sk)
{
	struct tcp_sock *tp = tcp_sk(sk);
	/*进入CWR后就不需要窗口撤消了,因此需要清除拥塞控制
		的慢启动阙值
	*/
	tp->prior_ssthresh = 0;
	if (inet_csk(sk)->icsk_ca_state < TCP_CA_CWR) {
		tp->undo_marker = 0;
		tcp_init_cwnd_reduction(sk);
		tcp_set_ca_state(sk, TCP_CA_CWR);
	}
}
\end{minted}

	\subsection{拥塞控制状态的处理及转换}

	\subsection{显式拥塞通知(ECN)}


