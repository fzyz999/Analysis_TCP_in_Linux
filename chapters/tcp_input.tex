\chapter{TCP输入}
\label{chapter:tcp_input}

    \minitoc

    \section{Linux内核网络数据接收流程概览}
        \begin{figure}[htb]        
            \centering
            \includegraphics[width=\textwidth]{images/Linux_Network_Receive.pdf}
        \end{figure}    
        正如上图所示,Linux内核在接收网络数据时分为以上5个层次。
        每个层次都有自己的功能,相互依赖,同时又独立成块。
        下面,我们来依次介绍这些层次。
\begin{enumerate}
\item[应用层]               对于应用层来说,提供多种接口来接收数据,包括read,recv,recvfrom.
\item[BSD Socket层]         这一层主要是在应用层进行进一步的处理。
\item[INET Socket层]            当被IP层的函数调用的时候,如果发现用户进程因正在操作传输控制块而将其锁定
                        就会将未处理的TCP段添加到后备队列中,而一旦用户进程解锁传输控制块,就会立即处理
                        后备队列,将TCP段处理之后的直接添加到接收队列中。
                        当被BSD Socket层的函数调用的时候,此时说明用户此时需要数据,那么我们就利用tcp\_recvmsg来
                        进行消息接收,当然,起初得判断队列是否为空,不为空才可以直接处理,为空就得进一步进行相关的处理。
\item[IP层]                 IP层则主要是在系统调度的时候,根据数据接收队列中是否由数据进行进一步的检验,
                        判断该包是不是发往该主机。然后进行数据包的重组(这是因为TCP数据有可能因为太大而被分片),
                        然后调用INET层的函数进行进一步的处理。
\item[硬件层]               这一部分主要是网卡接收到数据之后,进行发出请求中断,
                        然后调用相关函数进行处理,将接收到的数据放入到数据接收队列中。
                        当然,这时,那些没有经过FCS检验过的数据包这时候已经被抛弃了。
\end{enumerate}

        而在本文的分析中,我们主要关注的是TCP层的实现,故而我们主要分析INET Socket层。

        TCP传输控制块主要有三个关于接收的队列,接收队列,prequeue队列和后备队列。

        当启用tcp\_low\_latency(参见非核心部分讲解)时,TCP传输控制块在软中断中接收并处理TCP段,然后将其插入到
        接收队列中,等待用户进程从接收队列中获取TCP段后复制到用户空间中,最终删除并释放。

        而不启用tcp\_low\_latency这个选项的时候,则可以提高tcp协议栈的吞吐量及反应速度。TCP传输控制块在软中断中
        将TCP段添加到prequeue队列中,然后立即处理prequeue队列中的段。如果用户进程正在读取数据,则可以直接复制数据到用户空间
        的缓存区中,否则添加到接收队列中,然后从软中断返回。在多数情况下,有机会处理prequeue队列中的段,但只有当用户
        进程在进行recv类系统调用之前,才在软中断中复制数据到用户空间的缓存区。

        当然,在用户进程因操作传输控制块而将其锁定时,无论是否启用tcp\_low\_latency,都会将未处理的TCP段添加到后备队列中,一旦用户
        进程解锁传输控制块,就会立即处理后备队列,将TCP段处理之后添加到接收队列中。
    \section{自底向上调用与自顶向下调用}
        \subsection{自底向上处理}
            \subsubsection{\mintinline{C}{tcp_v4_rcv}}
                \label{TCPReceive:tcp_v4_rcv}
                正如上面的流程图所展示的样子,当IP层接收到报文,或将多个分片组装成一个完整的IP数据报之后,会调用
                传输层的接收函数,传递给传输层处理。
\begin{minted}[linenos]{C}
/*
Location:

    net/ipv4/tcp_ipv4.c

Function:

    TCP接收数据的入口。

Parameter:

    skb:从IP层传递过来的数据报。

*/
int tcp_v4_rcv(struct sk_buff *skb)
{
    const struct iphdr *iph;
    const struct tcphdr *th;
    struct sock *sk;
    int ret;
    struct net *net = dev_net(skb->dev);
    /*如果不是发往本机的就直接丢弃*/
    if (skb->pkt_type != PACKET_HOST)
        goto discard_it;

    /* Count it even if it's bad */
    TCP_INC_STATS_BH(net, TCP_MIB_INSEGS);
    /*
        如果TCP段在传输过程中被分片了,则到达本地后会在IP层重新组装。
        组装完成后,报文分片都存储在链表中。在此,需把存储在分片中的报文
        复制到SKB的线性存储区域中。如果发生异常,则丢弃该报文。
    */
    if (!pskb_may_pull(skb, sizeof(struct tcphdr)))
        goto discard_it;
\end{minted}

    关于pskb\_may\_pull更多的内容,请参见\ref{BSD:pskb_may_pull}.

\begin{minted}[linenos]{C}
    /*得到相关报文的头部*/
    th = tcp_hdr(skb);
    /*
        如果TCP的首部长度小于不带数据的TCP的首部长度,则说明TCP数据异常。
        统计相关信息后,丢弃。
    */
    if (th->doff < sizeof(struct tcphdr) / 4)
        goto bad_packet;
    /*
        检测整个TCP首部长度是否正常,如有异常,则丢弃。
    */  
    if (!pskb_may_pull(skb, th->doff * 4))
        goto discard_it;
\end{minted}

    关于pskb\_may\_pull更多的内容,请参见\ref{BSD:pskb_may_pull}.

\begin{minted}[linenos]{C}
    /* An explanation is required here, I think.
     * Packet length and doff are validated by header prediction,
     * provided case of th->doff==0 is eliminated.
     * So, we defer the checks. */
    /*
        验证TCP首部中的校验和,如校验和有误,则说明报文已损坏,统计相关信息后丢弃。
    */
    if (skb_checksum_init(skb, IPPROTO_TCP, inet_compute_pseudo))
        goto csum_error;
\end{minted}

    接下来,根据TCP首部中的信息来设置TCP控制块中的值,因为TCP首部中的值都是网络字节序
    的,为了便于后续处理直接访问TCP首部字段,在此将这些值转换为本机字节序后存储在TCP的
    私有控制块中。
\begin{minted}[linenos]{C}
    th = tcp_hdr(skb);
    iph = ip_hdr(skb);
    /* This is tricky : We move IPCB at its correct location into TCP_SKB_CB()
     * barrier() makes sure compiler wont play fool^Waliasing games.
     */
    memmove(&TCP_SKB_CB(skb)->header.h4, IPCB(skb),
        sizeof(struct inet_skb_parm));
    /*禁止编译器或底层做优化*/
    barrier();

    TCP_SKB_CB(skb)->seq = ntohl(th->seq);  //段开始序号
    TCP_SKB_CB(skb)->end_seq = (TCP_SKB_CB(skb)->seq + th->syn + th->fin +
                    skb->len - th->doff * 4);//
    TCP_SKB_CB(skb)->ack_seq = ntohl(th->ack_seq);//下一个等待发送的字节的序号
    TCP_SKB_CB(skb)->tcp_flags = tcp_flag_byte(th);
    TCP_SKB_CB(skb)->tcp_tw_isn = 0;
    TCP_SKB_CB(skb)->ip_dsfield = ipv4_get_dsfield(iph);
    TCP_SKB_CB(skb)->sacked  = 0;
\end{minted}
    
    接下来调用\mintinline{C}{__inet_lookup}在ehash或bhash散列表中根据地址和端口来查找
    传输控制块。如果在ehash中找到,则表示已经经历了三次握手并且已建立了连接,可以进行正常
    的通信。如果在bhash中找到,则表示已经绑定已经绑定了端口,处于侦听状态。如果在两个散列表中
    都查找不到,说明此时对应的传输控制块还没有创建,跳转到\mintinline{C}{no_tcp_socket}
    处处理。
\begin{minted}[linenos]{C}
lookup:
    sk = __inet_lookup_skb(&tcp_hashinfo, skb, th->source, th->dest);
    if (!sk)
        goto no_tcp_socket;
\end{minted}
    
    接下来继续处理。
\begin{minted}[linenos]{C}
process:
    //TIME_WAIT状态,主要处理释放连接
    if (sk->sk_state == TCP_TIME_WAIT)
        goto do_time_wait;
    //NEW_SYN_RECV状态,???
    if (sk->sk_state == TCP_NEW_SYN_RECV) {
        struct request_sock *req = inet_reqsk(sk);
        struct sock *nsk;

        sk = req->rsk_listener;
        if (unlikely(tcp_v4_inbound_md5_hash(sk, skb))) {
            reqsk_put(req);
            goto discard_it;
        }
        if (unlikely(sk->sk_state != TCP_LISTEN)) {
            inet_csk_reqsk_queue_drop_and_put(sk, req);
            goto lookup;
        }
        sock_hold(sk);
        nsk = tcp_check_req(sk, skb, req, false);
        if (!nsk) {
            reqsk_put(req);
            goto discard_and_relse;
        }
        if (nsk == sk) {
            reqsk_put(req);
        } else if (tcp_child_process(sk, nsk, skb)) {
            tcp_v4_send_reset(nsk, skb);
            goto discard_and_relse;
        } else {
            sock_put(sk);
            return 0;
        }
    }
    /*ttl小于给定的最小的ttl*/
    if (unlikely(iph->ttl < inet_sk(sk)->min_ttl)) {
        NET_INC_STATS_BH(net, LINUX_MIB_TCPMINTTLDROP);
        goto discard_and_relse;
    }
\end{minted}
    继续处理。
\begin{minted}[linenos]{C}
    //查找IPsec数据库,如果查找失败,进行相应处理.
    if (!xfrm4_policy_check(sk, XFRM_POLICY_IN, skb))
        goto discard_and_relse;
    //md5相关
    if (tcp_v4_inbound_md5_hash(sk, skb))
        goto discard_and_relse;

    nf_reset(skb);
    //过滤器
    if (sk_filter(sk, skb))
        goto discard_and_relse;
    /*
        设置SKB的dev为NULL,这是因为现在已经到了传输层,此时dev
        已经不再具有意义。
    */
    skb->dev = NULL;
    /*LISTEN状态*/
    if (sk->sk_state == TCP_LISTEN) {
        ret = tcp_v4_do_rcv(sk, skb);
        goto put_and_return;
    }

    sk_incoming_cpu_update(sk);
    /*
        在接收TCP段之前,需要对传输控制块加锁,以同步对传输控制块
        接收队列的访问。
    */
    bh_lock_sock_nested(sk);
    tcp_sk(sk)->segs_in += max_t(u16, 1, skb_shinfo(skb)->gso_segs);
    ret = 0;
    /*进程此时没有访问传输控制块*/
    if (!sock_owned_by_user(sk)) {
        /*prequeue可读取*/
        if (!tcp_prequeue(sk, skb))
            ret = tcp_v4_do_rcv(sk, skb);
    } else if (unlikely(sk_add_backlog(sk, skb,
                       sk->sk_rcvbuf + sk->sk_sndbuf))) {
            /*添加到后备队列成功*/
        bh_unlock_sock(sk);
        NET_INC_STATS_BH(net, LINUX_MIB_TCPBACKLOGDROP);
        goto discard_and_relse;
    }
    //解锁。
    bh_unlock_sock(sk);
\end{minted}

\begin{minted}[linenos]{C}
put_and_return:
    /*递减引用计数,当计数为0的时候,使用sk\_free释放传输控制块*/
    sock_put(sk);

    return ret;
/*
    处理没有创建传输控制块收到报文,校验错误,坏包的情况,给对端发送RST报文。
*/
no_tcp_socket:
    /*查找IPsec数据库,如果查找失败,进行相应处理.*/
    if (!xfrm4_policy_check(NULL, XFRM_POLICY_IN, skb))
        goto discard_it;

    if (tcp_checksum_complete(skb)) {
csum_error:
        TCP_INC_STATS_BH(net, TCP_MIB_CSUMERRORS);
bad_packet:
        TCP_INC_STATS_BH(net, TCP_MIB_INERRS);
    } else {
        tcp_v4_send_reset(NULL, skb);
    }
/*
    丢弃数据包。    
*/
discard_it:
    /* Discard frame. */
    kfree_skb(skb);
    return 0;

discard_and_relse:
    sock_put(sk);
    goto discard_it;
/*处理TIME_WAIT状态*/
do_time_wait:
    if (!xfrm4_policy_check(NULL, XFRM_POLICY_IN, skb)) {
        /*inet_timewait_sock减少引用计数*/      
        inet_twsk_put(inet_twsk(sk));
        goto discard_it;
    }
    if (tcp_checksum_complete(skb)) {
        inet_twsk_put(inet_twsk(sk));
        goto csum_error;
    }
\end{minted}
    关于inet\_twsk\_put更多的内容,请参见\ref{INET:inet_twsk_put}。

\begin{minted}[linenos]{c}
    //根据返回值进行相应处理
    switch (tcp_timewait_state_process(inet_twsk(sk), skb, th)) {
        case TCP_TW_SYN: {
            struct sock *sk2 = inet_lookup_listener(dev_net(skb->dev),
                                &tcp_hashinfo,
                                iph->saddr, th->source,
                                iph->daddr, th->dest,
                                inet_iif(skb));
            if (sk2) {
                inet_twsk_deschedule_put(inet_twsk(sk));
                sk = sk2;
                goto process;
            }
            /* Fall through to ACK */
        }
        case TCP_TW_ACK:
            tcp_v4_timewait_ack(sk, skb);
            break;
        case TCP_TW_RST:
            goto no_tcp_socket;
        case TCP_TW_SUCCESS:
            ;
    }
    goto discard_it;
}
\end{minted}

            \subsubsection{\mintinline{C}{tcp_prequeue}}
\begin{minted}[linenos]{C}
/* 
Location:

    net/ipv4/tcp_ipv4.c

Function:


    Packet is added to VJ-style prequeue for processing in process
    context(进程上下文), if a reader task is waiting. Apparently, this exciting
    idea (VJ's mail "Re: query about TCP header on tcp-ip" of 07 Sep 93)
    failed somewhere. Latency? Burstiness? Well, at least now we will
    see, why it failed. 8)8)                  --ANK

Parameter:

    sk:传输控制块
    skb:缓存区
*/
bool tcp_prequeue(struct sock *sk, struct sk_buff *skb)
{
	struct tcp_sock *tp = tcp_sk(sk);
	//如果开启了tcp_low_latency或者用户没有在读数据,直接返回
	if (sysctl_tcp_low_latency || !tp->ucopy.task)
		return false;
	/*
		数据包长度不大于tcp头部的长度,并且prequeue队列为空。
	*/
	if (skb->len <= tcp_hdrlen(skb) &&
	    skb_queue_len(&tp->ucopy.prequeue) == 0)
		return false;

	/* Before escaping RCU protected region, we need to take care of skb
	 * dst. Prequeue is only enabled for established sockets.
	 * For such sockets, we might need the skb dst only to set sk->sk_rx_dst
	 * Instead of doing full sk_rx_dst validity here, let's perform
	 * an optimistic check.
	 */
	if (likely(sk->sk_rx_dst))
		skb_dst_drop(skb);
	else
		skb_dst_force_safe(skb);
	/*
		将接收到的段添加到prequeue队列中,并更新prequeue队列消耗的内存。
	*/
	__skb_queue_tail(&tp->ucopy.prequeue, skb);
	tp->ucopy.memory += skb->truesize;
	/*
		如果prequeue消耗的内存超过接收缓存上限,则
		立刻处理prequeue队列上的段。
	*/
	if (tp->ucopy.memory > sk->sk_rcvbuf) {
		struct sk_buff *skb1;

		BUG_ON(sock_owned_by_user(sk));

		while ((skb1 = __skb_dequeue(&tp->ucopy.prequeue)) != NULL) {
			sk_backlog_rcv(sk, skb1);
			NET_INC_STATS_BH(sock_net(sk),
					 LINUX_MIB_TCPPREQUEUEDROPPED);
		}

		tp->ucopy.memory = 0;
	} else if (skb_queue_len(&tp->ucopy.prequeue) == 1) {	//队列长度为1
		wake_up_interruptible_sync_poll(sk_sleep(sk),
					   POLLIN | POLLRDNORM | POLLRDBAND);
		//不需要发送ack,则复位重新启动延迟确认定时器。
		if (!inet_csk_ack_scheduled(sk))
			inet_csk_reset_xmit_timer(sk, ICSK_TIME_DACK,
						  (3 * tcp_rto_min(sk)) / 4,
						  TCP_RTO_MAX);
	}
	return true;
}
\end{minted}
        \subsection{自顶向下处理}
            \subsubsection{\mintinline{C}{tcp_recvmsg}}
                我们一步一步对该函数进行分析。
\begin{minted}[linenos]{C}
/*
Location:

    net/ipv4/tcp.c

Function:

 *  This routine copies from a sock struct into the user buffer.
 *
 *  Technical note: in 2.3 we work on _locked_ socket, so that
 *  tricks with *seq access order and skb->users are not required.
 *  Probably, code can be easily improved even more.

Parameter:

	sk:传输控制块。
	msg:
	len:
	nonblock:是否阻塞
	flags:读取数据的标志。
	addrlen:
*/

int tcp_recvmsg(struct sock *sk, struct msghdr *msg, size_t len, int nonblock,
        int flags, int *addr_len)
{
    struct tcp_sock *tp = tcp_sk(sk);
    int copied = 0;
    u32 peek_seq;
    u32 *seq;
    unsigned long used;
    int err;
    int target;     /* Read at least this many bytes */
    long timeo;
    struct task_struct *user_recv = NULL;
    struct sk_buff *skb, *last;
    u32 urg_hole = 0;
    //如果只是为了接收来自套接字错误队列的错误,那就直接执行如下函数。
    if (unlikely(flags & MSG_ERRQUEUE))
        return inet_recv_error(sk, msg, len, addr_len);
    //不是特别理解,,????
    if (sk_can_busy_loop(sk) && skb_queue_empty(&sk->sk_receive_queue) &&
        (sk->sk_state == TCP_ESTABLISHED))
        sk_busy_loop(sk, nonblock);
\end{minted}
上述主要处理一些比较异常的情况,接下来就开始进行正常的处理了。
\begin{minted}[linenos]{C}
    /*
        在用户进程进行读取数据之前,必须对传输层进行加锁,这主要时为了在读的过程中,软中断
        操作传输层,从而造成数据的不同步甚至更为严重的不可预料的结果。
    */
    lock_sock(sk);
    //初始化错误码,Transport endpoint is not connected
    err = -ENOTCONN;
    /*
        如果此时只是处于LISTEN状态,表明尚未建立连接,
        此时不允许用户读取数据,只能返回。
    */
    if (sk->sk_state == TCP_LISTEN)
        goto out;
    /*
        获取阻塞读取的超时时间,如果进行非阻塞读取,则超时时间为0。
    */
    timeo = sock_rcvtimeo(sk, nonblock);

    /* Urgent data needs to be handled specially. 
        如果是要读取带外数据,则需要跳转到recv_urg进行处理。
    */
    if (flags & MSG_OOB)
        goto recv_urg;

\end{minted}
    继续处理。
\begin{minted}[linenos]{C}
	//被修复了
	if (unlikely(tp->repair)) {
		//	Operation not permitted
		err = -EPERM;
		//如果只是查看数据的话,就直接跳转到out处理
		if (!(flags & MSG_PEEK))
			goto out;
		//????
		if (tp->repair_queue == TCP_SEND_QUEUE)
			goto recv_sndq;
		//Invalid argument
		err = -EINVAL
		if (tp->repair_queue == TCP_NO_QUEUE)
			goto out;

		/* 'common' recv queue MSG_PEEK-ing */
	}
\end{minted}
	
	接下来进行数据复制。在把数据从接收缓存复制到用户空间的过程中,
	会更新当前已复制位置,以及段序号。如果接收数据,则会更新copied\_seq,
	但是如果只是查看数据而并不是从系统缓冲区移走数据,那么不能更新copied\_seq。
	因此,数据复制到用户空间的过程中,区别接收数据还是查看数据是根据是否更新copied\_seq,
	所以这里时根据接收数据还是查看来获取要更新标记的地址,而后面的复制操作就完全不关心时接收
	还是查看。
	
	最后一行调用相关函数根据是否设置了MSG\_WAITALL来确定本次调用需要接收数据的长度。如果设置
	了MSG\_WAITALL标志,则读取数据长度为用户调用时输入的参数len。
\begin{minted}[linenos]{C}
    //被修复了???啥意思》?》
    if (unlikely(tp->repair)) {
        //  Operation not permitted
        err = -EPERM;
        //如果只是查看数据的话,就直接跳转到out处理
        if (!(flags & MSG_PEEK))
            goto out;
        //????
        if (tp->repair_queue == TCP_SEND_QUEUE)
            goto recv_sndq;
        //Invalid argument
        err = -EINVAL
        if (tp->repair_queue == TCP_NO_QUEUE)
            goto out;

        /* 'common' recv queue MSG_PEEK-ing */
    }
\end{minted}
    
    接下来进行数据复制。在把数据从接收缓存复制到用户空间的过程中,
    会更新当前已复制为止,以及段序号。如果接收数据,则会更新copied\_seq,
    但是如果只是查看数据而并不是从系统缓冲区移走数据,那么不能更新copied\_seq。
    因此,数据复制到用户空间的过程中,区别接收数据还是查看数据是根据是否更新copied\_seq,
    所以这里时根据接收数据还是查看来获取要更新标记的地址,而后面的复制操作就完全不关心时接收
    还是查看。
    
    最后一行调用相关函数根据是否设置了MSG\_WAITALL来确定本次调用需要接收数据的长度。如果设置
    了MSG\_WAITALL标志,则读取数据长度为用户调用时输入的参数len。
\begin{minted}[linenos]{C}
    seq = &tp->copied_seq;
    if (flags & MSG_PEEK) {
        peek_seq = tp->copied_seq;
        seq = &peek_seq;
    }

    target = sock_rcvlowat(sk, flags & MSG_WAITALL, len);
\end{minted}

    接下来通过urg\_data和urg\_seq来检测当前是否读取到带外数据。如果在
    读带外数据,则终止本次正常数据的读取。否则,如果用户进程有信号待处理,
    则也终止本次的读取。
\begin{minted}[linenos]{C}
    do {
        u32 offset;

        /* Are we at urgent data? Stop if we have read anything or have SIGURG pending. */
        if (tp->urg_data && tp->urg_seq == *seq) {
            if (copied)
                break;
            if (signal_pending(current)) {
                copied = timeo ? sock_intr_errno(timeo) : -EAGAIN;
                break;
            }
        }
\end{minted}

    接下来首先获取一个缓存区。
\begin{minted}[linenos]{C}
		/* Next get a buffer. */
		//获取下一个要读取的的段。
		last = skb_peek_tail(&sk->sk_receive_queue);
		skb_queue_walk(&sk->sk_receive_queue, skb) {
			last = skb;
			/* Now that we have two receive queues this
			 * shouldn't happen.
				如果接收队列中的段序号比较大,则说明也获取不到下一个待获取的段,
				这样也只能接着处理prequeue或后备队列,实际上这种情况不应该发生。
			 */
			if (WARN(before(*seq, TCP_SKB_CB(skb)->seq),
				 "recvmsg bug: copied %X seq %X rcvnxt %X fl %X\n",
				 *seq, TCP_SKB_CB(skb)->seq, tp->rcv_nxt,
				 flags))
				break;
			/*
				到此,我们已经获取了下一个要读取的数据段,计算该段开始读取数据的偏移位置,
				当然,该偏移值必须在该段的数据长度范围内才有效。
			*/
			offset = *seq - TCP_SKB_CB(skb)->seq;	
			/*
				由于SYN标志占用了一个序号,因此如果存在SYN标志,则需要调整
				偏移。由于偏移offset为无符号整型,因此,不会出现负数的情况。
			*/
			if (TCP_SKB_CB(skb)->tcp_flags & TCPHDR_SYN)
				offset--;
			/*
				只有当偏移在该段的数据长度范围内,才说明待读的段才是有效的,因此,接下来
				跳转到found_ok_skb标签处读取数据。
			*/
			if (offset < skb->len)
				goto found_ok_skb;
			/*
				如果接收到的段中有FIN标识,则跳转到found_fin_ok处处理。
			*/
			if (TCP_SKB_CB(skb)->tcp_flags & TCPHDR_FIN)
				goto found_fin_ok;
			WARN(!(flags & MSG_PEEK),
			     "recvmsg bug 2: copied %X seq %X rcvnxt %X fl %X\n",
			     *seq, TCP_SKB_CB(skb)->seq, tp->rcv_nxt, flags);
		}
\end{minted}

    只有在读取完数据后,才能在后备队列不为空的情况下区处理接收到后备队列中的TCP段。
    否则终止本次读取。

    由于是因为用户进程对传输控制块进行的锁定,将TCP段缓存到后备队列,故而,一旦用户进程释放传输控制块就应该立即处理后备队列。
    处理后备队列直接在release\_sock中实现,以确保在任何时候解锁传输控制块时能立即处理后备队列。

\begin{minted}[linenos]{C}
        /* Well, if we have backlog, try to process it now yet. */

        if (copied >= target && !sk->sk_backlog.tail)
            break;
\end{minted}

	当接收队列中可以读取的段已经读完,在处理prequeue或后备队列之前,我们需要先检查是否会存在一些
	异常的情况。如果存在这类情况,就需要结束这次读取,返回前当然还顺便检测后备队列是否存在数据,如果有
	则还需要处理。

\begin{minted}[linenos]{C}
        if (copied) {
            /*
                检测条件:
                1.有错误发生
                2.TCP处于CLOSE状态
                3.shutdown状态
                4.收到信号
                5.只是查看数据
            */          
            if (sk->sk_err ||
                sk->sk_state == TCP_CLOSE ||
                (sk->sk_shutdown & RCV_SHUTDOWN) ||
                !timeo ||
                signal_pending(current))
                break;
        } else {
            /*检测TCP会话是否即将终结*/
            if (sock_flag(sk, SOCK_DONE))
                break;
            /*如果有错误,返回错误码*/
            if (sk->sk_err) {
                copied = sock_error(sk);
                break;
            }
            /*如果是shutdown,返回*/
            if (sk->sk_shutdown & RCV_SHUTDOWN)
                break;
            /*
                如果TCP状态为CLOSE,而套接口不再终结状态,则进程可能
                在读取一个从未建立连接的套接口,因此,返回相应的错误码。
            */
            if (sk->sk_state == TCP_CLOSE) {
                if (!sock_flag(sk, SOCK_DONE)) {
                    /* This occurs when user tries to read
                     * from never connected socket.
                     */
                    copied = -ENOTCONN;
                    break;
                }
                break;
            }
            /*
                未读到数据,且是非阻塞读取,则返回错误码Try again。
            */
            if (!timeo) {
                copied = -EAGAIN;
                break;
            }
            /*检测是否收到数据,并返回相应的错误码*/
            if (signal_pending(current)) {
                copied = sock_intr_errno(timeo);
                break;
            }
        }
\end{minted}

    进一步处理。
\begin{minted}[linenos]{C}
        //检测是否有确认需要立即发送
        tcp_cleanup_rbuf(sk, copied);
        /*
            在未启用慢速路径的情况下,都会到此检测是否
            需要处理prepare队列。
        */      
        if (!sysctl_tcp_low_latency && tp->ucopy.task == user_recv) {
            /* Install new reader 
                如果此次是第一次检测处理prepare队列的话,则需要设置正在读取的
                进程标识符、缓存地址信息,这样当读取进程进入睡眠后,ESTABLISHED
                状态的接收处理就可能把数据复制到用户空间。          
            */
            if (!user_recv && !(flags & (MSG_TRUNC | MSG_PEEK))) {
                user_recv = current;
                tp->ucopy.task = user_recv;
                tp->ucopy.msg = msg;
            }
            //更新当前可以使用的用户缓存的大小。
            tp->ucopy.len = len;
            //进行报警信息处理。
            WARN_ON(tp->copied_seq != tp->rcv_nxt &&
                !(flags & (MSG_PEEK | MSG_TRUNC)));

            /* Ugly... If prequeue is not empty, we have to
             * process it before releasing socket, otherwise
             * order will be broken at second iteration.
             * More elegant solution is required!!!
             *
             * Look: we have the following (pseudo)queues:
             *
             * 1. packets in flight
             * 2. backlog
             * 3. prequeue
             * 4. receive_queue
             *
             * Each queue can be processed only if the next ones
             * are empty. At this point we have empty receive_queue.
             * But prequeue _can_ be not empty after 2nd iteration,
             * when we jumped to start of loop because backlog
             * processing added something to receive_queue.
             * We cannot release_sock(), because backlog contains
             * packets arrived _after_ prequeued ones.
             *
             * Shortly, algorithm is clear --- to process all
             * the queues in order. We could make it more directly,
             * requeueing packets from backlog to prequeue, if
             * is not empty. It is more elegant, but eats cycles,
             * unfortunately.
             */
            /*
                如果prequeue队列不为空,则跳转到do\_orequeue标签处处理prequeue队列。            
            */
            if (!skb_queue_empty(&tp->ucopy.prequeue))
                goto do_prequeue;

            /* __ Set realtime policy in scheduler __ */
        }
\end{minted}
    
    继续处理,如果读取完数据,则调用release\_sock来解锁传输控制块,主要用来处理后备队列,完成
    后再调用lock\_sock锁定传输控制块。在调用release\_sock的时候,进程有可能会出现休眠。

    如果数据尚未读取,且是阻塞读取,则进入睡眠等待接收数据。这种情况下,tcp\_v4\_do\_rcv处理TCP段时可能会把
    数据直接复制到用户空间。
\begin{minted}[linenos]{C}
        if (copied >= target) {
            /* Do not sleep, just process backlog. */
            release_sock(sk);
            lock_sock(sk);
        } else {
            sk_wait_data(sk, &timeo, last);
        }
\end{minted}

\begin{minted}[linenos]{C}
        if (user_recv) {
            int chunk;

            /* __ Restore normal policy in scheduler __ */
            /*
                更新剩余用户空间长度以及已经复制到用户空间的数据长度
            */
            chunk = len - tp->ucopy.len;
            if (chunk != 0) {
                NET_ADD_STATS_USER(sock_net(sk), LINUX_MIB_TCPDIRECTCOPYFROMBACKLOG, chunk);
                len -= chunk;
                copied += chunk;
            }
            /*
                如果接收到接收队列中的数据已经全部复制到用户进程空间,
                但prequeue队列不为空,则需要继续处理prequeue队列,并
                更新剩余的用户空间的长度和已经复制到用户空间的数据长度。                
            */
            if (tp->rcv_nxt == tp->copied_seq &&
                !skb_queue_empty(&tp->ucopy.prequeue)) {
do_prequeue:
                tcp_prequeue_process(sk);

                chunk = len - tp->ucopy.len;
                if (chunk != 0) {
                    NET_ADD_STATS_USER(sock_net(sk), LINUX_MIB_TCPDIRECTCOPYFROMPREQUEUE, chunk);
                    len -= chunk;
                    copied += chunk;
                }
            }
        }
\end{minted}
        处理完prequeue队列后,如果有更新copied\_seq,且只是查看数据,则需要更新peek\_seq。
        然后继续获取下一个段进行处理。
\begin{minted}[linenos]{C}
        if ((flags & MSG_PEEK) &&
            (peek_seq - copied - urg_hole != tp->copied_seq)) {
            net_dbg_ratelimited("TCP(%s:%d): Application bug, race in MSG_PEEK\n",
                        current->comm,
                        task_pid_nr(current));
            peek_seq = tp->copied_seq;
        }
        continue;
\end{minted}
        
        
\begin{minted}[linenos]{C}
    found_ok_skb:
        /* Ok so how much can we use? */
        /*
            获取该可读取段的数据长度,在前面的处理中
            已由TCP序号得到本次读取数据在该段中的偏移。
        */
        used = skb->len - offset;
        if (len < used)
            used = len;

        /* Do we have urgent data here? 
            如果该段中包含带外数据,则获取带外数据在该段中的偏移。
            如果偏移在该段可读的范围内,则表示带外数据有效。
            进而
                如果带外数据偏移为0,则说明目前需要的数据正是带外数据,
            且带外数据不允许放入到正常的数据流中,即在普通的数据数据
            流中接受带外数据,则需要调整读取正常数据流的一些参数,如已
            读取数据的序号、正常数据的偏移等。最后,如果可读数据经过调
            整之后为0,则说明没有数据可读,跳过本次读数据过程到skip_copy处
            处理。
                如果带外数据偏移不为0,则需要调整本次读取的正常长度直到读到带外
            数据为止。
        */
        if (tp->urg_data) {
            u32 urg_offset = tp->urg_seq - *seq;
            if (urg_offset < used) {
                if (!urg_offset) {
                    if (!sock_flag(sk, SOCK_URGINLINE)) {
                        ++*seq;
                        urg_hole++;
                        offset++;
                        used--;
                        if (!used)
                            goto skip_copy;
                    }
                } else
                    used = urg_offset;
            }
        }
\end{minted}
    
    接下来处理读取数据的情况。
\begin{minted}[linenos]{C}
        if (!(flags & MSG_TRUNC)) {
            /*
                调用skb_copy_datagram_msg来将数据复制到用户空间
                并且根据返回的值来判断是否出现了错误。
            */          
            err = skb_copy_datagram_msg(skb, offset, msg, used);
            if (err) {
                /* Exception. Bailout! */
                if (!copied)
                    copied = -EFAULT;
                break;
            }
        }
        /*
            调整读正常数据流的一些参数,如已读取数据的序号、已读取
            数据的长度,剩余的可以使用的用户空间缓存大小。如果是截短,
            则通过调整这些参数,多余的数据就默默被丢弃了。
        */
        *seq += used;
        copied += used;
        len -= used;
        /*tcp_rcv_space_adjust调整合理的TCP接收缓存的大小*/
        tcp_rcv_space_adjust(sk);
\end{minted}

    
\begin{minted}[linenos]{C}
skip_copy:
        /*
            如果已经完成了对带外数据的处理,则将带外数据标志清零,
            设置首部预测标志,下一个接收到的段,就又可以通过首部预测执行快速
            路径还是慢速路径了。
        */
        if (tp->urg_data && after(tp->copied_seq, tp->urg_seq)) {
            tp->urg_data = 0;
            tcp_fast_path_check(sk);
        }
        /*
            如果该段还有数据没有读取(如带外数据),则只能继续处理该段,而不能将
            该段从接收队列中删除。
        */
        if (used + offset < skb->len)
            continue;
        /*如果发现段中存在FIN标志,则跳转到found\_fin\_ok标签处处理*/
        if (TCP_SKB_CB(skb)->tcp_flags & TCPHDR_FIN)
            goto found_fin_ok;
        /*
            如果已经读完该段的全部数据,且不是查看数据,则可以将该段从接收队列中
            删除,然后继续处理后续的段。        
        */
        if (!(flags & MSG_PEEK))
            sk_eat_skb(sk, skb);
        continue;
    /*
        由于FIN标志占一个序号,因此当前读取的序号需要递增。
        如果已经读完该段的全部数据并且不是查看数据,则可以将该段从
        接收队列中删除。然后就可以退出了,无需处理后续的段了。
    */
    found_fin_ok:
        /* Process the FIN. */
        ++*seq;
        if (!(flags & MSG_PEEK))
            sk_eat_skb(sk, skb);
        break;
    } while (len > 0);
\end{minted}

    
\begin{minted}[linenos]{C}
    if (user_recv) {
        //prequeue队列非空
        if (!skb_queue_empty(&tp->ucopy.prequeue)) {
            int chunk;

            tp->ucopy.len = copied > 0 ? len : 0;
            //处理prequeue队列。
            tcp_prequeue_process(sk);
            /*
                如果在处理prequeue队列的过程中又有一部分数据复制到用户空间,则需要调整剩余的
                可用用户空间缓存大小和已读取数据的序号。
            */
            if (copied > 0 && (chunk = len - tp->ucopy.len) != 0) {
                NET_ADD_STATS_USER(sock_net(sk), LINUX_MIB_TCPDIRECTCOPYFROMPREQUEUE, chunk);
                len -= chunk;
                copied += chunk;
            }
        }
        /*
            最后清零ucopy.task和ucopy.len,表示用户当前没有读取数据。这样当处理prequeue队列时,
            不会将数据复制到用户空间,因为只有在未启用tcp_low_latency的情况下,用户进程主动读取
            时,才有机会将数据直接复制到用户空间。
        */
        tp->ucopy.task = NULL;
        tp->ucopy.len = 0;
    }
\end
    
    在完成读取数据后,需再次检测是否有必要立即发送ACK段,并根据情况确定是否发送ACK段。在返回之前需解锁传输控制块。
    返回已经读取的字节数。
\begin{minted}[linenos]{C}
    /* According to UNIX98, msg_name/msg_namelen are ignored
     * on connected socket. I was just happy when found this 8) --ANK
     */

    /* Clean up data we have read: This will do ACK frames. */
    tcp_cleanup_rbuf(sk, copied);

    release_sock(sk);
    return copied;
\end{minted}
    接下来,进行一些其他情况的处理。
\begin{minted}[linenos]{C}
/*
    如果在读取的过程中遇到l错误,就会跳转到此,解锁传输层然后返回错误码。
*/
out:
    release_sock(sk);
    return err;
/*
    如果是接收带外数据,则调用tcp_recv_urg接收。
*/
recv_urg:
    err = tcp_recv_urg(sk, msg, len, flags);
    goto out;
/*
??这一步是干什么的呢?
*/
recv_sndq:
    err = tcp_peek_sndq(sk, msg, len);
    goto out;
}
\end{minted}

